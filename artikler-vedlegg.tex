\chapter{Vedlegg: Oversikt artikler}

\begin{landscape}
  \begin{table}
    \tiny
   {\sffamily
    \begin{tabularx}{\paperwidth}{
        l
        >{\raggedright\arraybackslash}X
        >{\raggedright\arraybackslash}X
        >{\raggedright\arraybackslash}X
        >{\raggedright\arraybackslash}X
        >{\raggedright\arraybackslash}X}
      \textbf{Artikkel} &
      \textbf{Tittel} &
      \textbf{Tittel} &
      \textbf{Tittel} &
      \textbf{Tittel} &
      \textbf{Tittel} \\
      &
      Cancer survivors experiences of humour while navigating through challenging landscapes --- a socio-narrative approach. &
      More than trivial. Strategies for using humour in palliative care. &
      From critical care to comfort care: the sustaining value of humour. &
      Humour in health care interactions: a risk worth taking. &
      A time to weep and a time to laugh: humour in the nurse-patient relationship in an adult cancer care setting.
      \\
      \\
      &
      \textbf{Forfatter} &
      \textbf{Forfatter} &
      \textbf{Forfatter} &
      \textbf{Forfatter} &
      \textbf{Forfatter}
      \\
      &
      Roaldsen, B.L., Sørlie, T. \&{} Lorem, G.F. &
      Dean, R.A.K. \&{} Gregory D.M. &
      Dean, R.A.K. \&{} Major, J.E. &
      McCreaddie, M. \&{} Payne, S. &
      Tanay, M.A., Wiseman T., Roberts J. \&{} Ream E.
      \\
      \\
      &
      \textbf{Utgitt i} &
      \textbf{Utgitt i} &
      \textbf{Utgitt i} &
      \textbf{Utgitt i} &
      \textbf{Utgitt i}
      \\
      &
      Scandinavian Journal of Caring Sciences (2014) &
      Cancer Nursing (2005) &
      Journal of Clinical Nursing (2007) &
      Health Expectations (2011) &
      Support Care in Cancer (2013)
      \\
      \\
      \textbf{Perspektiv} &
      Pasient &
      Pasient, familie og helsepersonell &
      Pasient, familie og helsepersonell &
      Pasient og sykepleier &
      Pasient og sykepleier
      \\
      \\
      \textbf{Problem/Hensikt} &
      Artikkelen ønsker å belyse hvordan kreftfrie pasienter opplevde og
      evaluerte betydningen av humor i dagliglivet, fra diagnosetidspunkt og
      gjennom sykdomsforløpet. Den ønsker å få en bedre forståelse for hvordan
      humor brukes som en mestringsstrategi i en livstruende situasjon.
      &
      Artikkelen ønsker å beskrive i hvilke situasjoner humor og latter
      oppstår, hvilken funksjon det har, og identifisere situasjoner hvor humor
      og latter er upassende. Artikkelen ønsker også å se på hvilke faktorer
      som påvirker bruken av humor og hvilke strategier som anbefales for bruk
      av humor i palliativ pleie.
      &
      Artikkelen ønsker å illustrere verdien humor har i team-arbeid og
      pasientbehandling, uavhengig av ulike settinger.
      &
      Artikkelen undersøke pasientenes perspektiv på bruk av humor i
      helsevesenet. Den ser på humorens motsatte effekt i bruk mellom pasient
      og sykepleier, og hvor man antar at en medvirkende faktor er
      sykepleiernes risikobruk av humor.
      &
      Artikkelen undersøker bruken av humor gjennom samhandlingen mellom
      pasient og sykepleier i en kreftavdeling for voksne. Den undersøker
      hvordan både pasient og sykepleier bruker humor som en mestringsstrategi
      i en vanskelig situasjon.
      \\
      \\
      \textbf{Metode} &
      Kvalitativ metode &
      Kvalitativ metode, klinisk etnografi &
      Kvalitativ metode, klinisk etnografi &
      Kvalitativ metode &
      Kvalitativ metode, klinisk etnografi
      \\
      \\
      \textbf{Resultat} &
      Pasientene beskrev humor som en hjelp til å takle vanskelige situasjoner,
      og relaterte plager til tross for svingninger i sykdomsforløpet. Humor
      ble brukt som et hjelpemiddel for å takle situasjonen de var i, og for å
      forebygge at sykdommen overskygget hele deres tilværelse. Bruken av humor
      ble relatert særlig til tre hovedkategorier: <<takle en livstruende
      situasjon>>, <<samhold og kommunikasjon>> og <<å leve med situasjonen>>.
      &
      Bruk av mottakelse av humor avhenger av faktorer som omstendigheter,
      personlighet, etnisitet, kjønn og stressnivå. Man fant at humor som regel
      oppstod spontant, og at man bare hadde en intuisjon om når det passet og
      ikke. Man lærte for å lese pasientene og deres respons på humor. Humor
      ble brukt til å bygge relasjoner, lette på stemningen og for å takle
      stressende situasjoner og vanskelige følelser.
      &
      Artikkelen viser at bruk av humor kan ha positiv effekt på samarbeid,
      redusere spenning, utvikle følelsesmessig fleksibilitet og til å
      menneskeliggjøre helsevesenet både for helsepersonell og bruker.
      &
      Pasientene sier de setter pris på bruk av humor, og at de gjenkjenner
      humor også i subtile og nyanserte former. Pasientene ønsker at
      helsepersonell både skal oppfordre til bruk av humor, samt og svare på
      humor.
      &
      Humor kan brukes til å skape gode relasjoner mellom sykepleier og
      pasient. Pasientene brukte med vilje humor i situasjoner de så var
      stressende for sykepleier. Dette gjorde de for å lette på og gjøre
      situasjonen letter for sykepleieren. Mens pasientene mente at bruk av
      humor var en god egenskap, stilte sykepleierne selv spørsmål til egen
      profesjonalitet ved bruk av humor. En konstant overveielse og refleksjon
      over bruk av humor i sykepleien sikrer at humor blir brukt
      hensiktsmessig.
      \\
    \end{tabularx}}
    \label{tabell.artikler.vedlegg}
  \end{table}
\end{landscape}
