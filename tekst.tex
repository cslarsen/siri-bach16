\chapter{Innledning}

Kreft er en sykdom som rammer mange mennesker. Det er en alvorlig og
livstruende sykdom disse menneskene møter. Det innebærer mange utfordringer og
følelser som angst, uro og bekymringer for egen livssituasjon. Forandring som
dette påvirker livskvaliteten og livet generelt. Flere opplever at de selv
stenger omverden ute, eller at andre trekker seg unna når de har fått en
kreftdiagnose (Rustøen, 2008; 39-47). Sorgtunge ansikt og lutende skuldre er
noe flere forbinder med kreft. Kreft er en sykdom som ikke er komisk. Det er
forventet at man skal gråte av dette, ikke le (Tyrdal 2002; 161-162) I dag er
det en større gruppe en tidligere som lever lenger med kreft, dette medfører at
pasientene har andre behov enn tidligere. Helsepersonell har ikke bare oppgaver
knyttet til pasientens fysiske helse, men også oppgaver som fokuserer på å
gjøre livet til den enkelte bedre, ved å øke livskvaliteten. Selv om en person
får en kreftdiagnose, så mister de ikke sin humoristiske sans. De er det samme
mennesket ennå, selv om de kanskje har litt endrede behov enn tidligere. Bruk
av humor i kreftomsorgen kan ha en viktig betydning for pasienten, dette kan
hjelpe til å bevare følelsen av å være et levende menneske (Tyrdal 2002; 163)
Vi har i denne oppgaven valgt å hente mye av teorien fra bøkene
\textit{Kreftsykepleie} og \textit{Humor og helse}. Dette er bøker som gir oss
både sykepleier og pasientperspektivet ved samhandling dem imellom.

\section{Bakgrunn for valg av tema}

Etter hjerte- og karsykdommer er kreft den vanligste dødsårsaken i Norge. Hvert
år dør i overkant av 10 000 mennesker av kreft. I 2014 fikk 31 651 nordmenn en
kreftdiagnose\index{kreftdiagnose}, og 10 971 døde. Risikoen for å få kreft
øker med alderen, men rammer alle aldersgrupper\index{aldersgrupper}. (kreftforeningen.no).

Ordet humor er vanskelig å definere, og man har ikke lyktes i å lage en felles
teori om humor. Grunnen til dette er at humor er noe personlig og avhenger av
individet og situasjonen det befinner seg i (Tyrdal 2002).

Som sykepleierstudenter og fra tidligere erfaring har vi merket oss at humor
kan ha stor positiv effekt på samarbeidet med pasienter. Vi vet at humor er
kontaktskapende når den brukes til å bygge relasjoner. Vi opplever at det er
med på å åpne opp for et godt samarbeid, og å skape en god relasjon mellom
sykepleier og pasient. Vi har også erfart at man gjerne blir litt mer
tilbakeholden til bruk av humor ved alvorlig sykdom, og gjerne til den siste
fasen av livet. Vi ønsket derfor å se litt nærmere på hvordan og i hvilke
situasjoner humor blir brukt, og hvilken effekt det har i møte med pasienter
som lider av kreftsykdom.

\section{Problemformulering}

Kreft er en alvorlig og livstruende sykdom, og mange sykepleiere vegrer seg
derfor mot å bruke humor i samhandling med kreftpasienter. Mange er usikre på
humorens virkning på pasienter, eller er <<redde>> for at humoren kan bli
missforstått og derfor ha motsatt virkning. Vi ønsker å finne ut hvordan humor
blir mottatt av pasienter med kreft. Det er viktig at vi som sykepleiere vet
hvordan vi skal bruke humor i kommunikasjon med pasienter, og hvilken virkning
humoren kan ha i samhandling med kreftsyke pasienter. Samt at vi vet noe om
fordeler og ulemper ved bruk av humor.

Det er lite litteratur og forskning på bruk av humor i sykepleie, og det er
vanskelig å vite noe om hvordan det fungerer. Man kan oppleve at det blir tatt
i bruk på feil måte, eller at man er redd for å bruke humor i det hele. For å
vite hvordan man kan bruke humor, må man vite noe om hva pasienter opplever som
positivt og negativt ved humor bruk

\section{Hensikt}

Oppgavens faglige hensikt er først og fremst å se på hvordan humor blir
praktisert i sykepleien. Vi ønsker å se på hvordan humor kan brukes i relasjon
mellom pasient og sykepleier. I tillegg ønsker vi å ha særlig fokus på humor i
kommunikasjon og som mestringstrategi.

Formelt sett representerer oppgaven avsluttende eksamen for Bachelor i
Sykepleie ved UIS, kull 2012-2016.

\section{Begrepsavklaring}
\todo{Hvilke begrep skal vi forklare?}

\chapter{Teori}
\section{Kreft}

Kreft har blitt en folkesykdom som rammer en stadig større del av den norske
befolkningen. Den medisinske utviklingen fører til at flere lever med sin
kreftsykdom. Ordet kreft utløser hos flere sterke assosiasjoner til døden. Mot
en snikende død, hvor fienden er usynlig og kommer innenfra ens egen organisme.
(livet med kreft s. 8) Flere og flere overlever kreft, men det er fremdeles en
sykdom som skremmer oss. Grunnen til dette kan være fordi den rammer så mange
og at de som blir rammet får livet sitt dramatisk forandret eller forkortet.
Det er noe skummelt med en sykdom som oppstår ved at kroppens egne mekanismer
løper løpsk og kommer ut av kontroll, ofte uten at man merker det. (livet med
kreft s. 8) Flere sier at de ikke ønsker å bli assosiert med kreft i
offentligheten, de ønsker ikke å ha en identitet som er knyttet til kreft.
Dette forteller litt om de forestillinger som finnes om kreft blant mennesker
og de fordommer som eksisterer. Mange har opplevd at om de snakker om sin kreft
i offentligheten, «blir» de sin kreft. Det er få andre sykdommer det er slik
med. (livet med kreft s.10)

\subsection{Livskvalitet}

Livskvalitet rommer både fysiske, psykiske og sosiale aspekt ved livet. En
norsk psykolog har utarbeidet en definisjon som hevder at et menneske har det
godt og har god livskvalitet, dersom man er i aktivitet, har samhørighet med
andre, har en god selvfølelse og har en grunnstemning av glede. Det som gjør
begrepet livskvalitet viktig innen kreftomsorgen, er at kreft rammer hele
mennesket. Vi takler det å bli syk på forskjellige måter. Samme sykdom kan gi
ulike reaksjoner, som kan avhenge av fase i livet, det sosiale nettverk og
tidligere erfaringer. (Rustøen, 2008: 39-40) Håp har stor betydning for
livskvaliteten, både fortid, nåtid og fremtid. Håp knyttet til fremtiden er en
viktig faktor i all pleie og omsorg. Det at fremtiden er usikker, er en av
grunnene til at kreftsykdom medfører lidelse. Ved for mye lidelse kan håpet
svekkes. Håp er en følelse som kan påvirkes av holdning og atmosfære. Derfor
kan sykepleiers atferd og tilstedeværelse spille en rolle for pasienten. Det er
viktig å ha tro på seg selv og sine evner for å kunne takle hverdagen best
mulig. Ens eget selvbilde, troen på at en kan tilpasse seg eventuelle
begrensninger i livet, evnen til å løse problemer er sentrale områder. Det å
sette ord på og bli bevisst på følelsesmessige reaksjoner i forhold til
kreftbehandlingen. Å se betydningen av de man har rundt seg, som familie,
venner, helsevesen og videre. Familie og venner er de to viktigste kildene for
håp for mennesker med kreft, særlig det at man kan få hjelp av dem. Ofte blir
det understreket at håp er en viktig faktor for å kunne handle. Det gir energi
og krefter til å kjempe og det blir en drivkraft for å komme seg videre.
(Rustøen, 2008; 40-41) Det å finne mening i sin situasjon, er en viktig faktor
for å kunne bearbeide en krise. O`Connor (1990) undersøkte hvordan
kreftpasienter søkte mening. Det han fant ut var at de vesentlige
forutsetningene var sosial støtte og personlig tro. Det var viktig med
forståelse for diagnose, konsekvenser, vilje til å tenke gjennom livet sitt,
endring i synet på seg selv og andre mennesker, evne til å meste det å leve med
kreft og håp. (Rustøen, 2008;41) Som sykepleier kan man hjelpe pasienten til å
endre holdninger knyttet til seg selv og sin sykdom. Lidelse kan settes inn i
en annen kontekst å på denne måten blir ens forventninger mer i samsvar med det
man erfarer. Dette gir pasienten mulighet til å leve et meningsfylt liv, selv
med de begrensninger sykdom kan gi. Travelbee gjør rede for metoder som
sykepleiere kan bruke for å hjelpe pasienten til opplevelse av meningsfull
tilværelse. En av disse metodene går ut på å hjelpe pasienten til å innse at
alle mennesker opplever lidelse gjennom livet. Ved å avfinne seg med dette, kan
man komme over det første sjokket og starte med å bearbeide krisen. (Rustøen,
2008; 42) Vi ser ofte at sykdom fører til isolasjon. Enten ved at pasienten
stenger seg ute fra omverden, eller at de rundt pasienten trekker seg unna. En
undersøkelse som ble utført i Norge viser at flesteparten av kreftpasienter
opplevde en stor grad av støtte og hjelp fra dem rundt seg, i forbindelse med
sykdom. Familie er ofte den viktigste kilden til støtte, de kan gi støtte og
håp når pasienten er i ferd med å gi opp og være en påminnelse på hvordan livet
var før sykdom. Ved innleggelse kan pleiepersonell og medpasienter spille en
viktig rolle for pasienten. (Rustøen, 2008;42) Flere mennesker som har gått
gjennom en kreftbehandling, hevder at behandlingen var en trussel mot deres
selvbilde og egenverd. Sykdom kan føre til endret syn på seg selv, et endret
selvbilde. Selvbilde blir ofte referert til utseende, men for kreftpasienter er
det også viktig å føle seg som et helt menneske, fungere normalt og å oppleve
integritet. Å føle seg svak, trett og slapp, samt ha en endret kropp eller
endrede kroppsfunksjoner, kan true en persons selvbilde. Responsen man får fra
de rundt seg på kreft, er viktig for pasientens identitet. Pårørende må huske
på at pasienten er den samme som før sykdommen, men at de kan ha behov for
annen eller mer støtte enn tidligere. Dette kan sykepleier hjelpe å ivareta.
(Rustøen, 2008; 42-43)

\subsection{Krise}

Kreftpasienter og deres pårørende står overfor store utfordringer, som ofte
innebærer angst, uro og bekymring for egen livssituasjon. Pasienten kan oppleve
en traumatisk krise. Det som kjennetegner en krise, er at personens tidligere
erfaringer og reaksjoner ikke er tilstrekkelig for å mestre og forstå
situasjonen. Kriser har forskjellig styrke og intensitet og man opplever
forskjellig grad av problemløsning. (Reitan, 2008 s. 48) Helsepersonell må
kjenne til psykiske reaksjoner ved alvorlige og livstruende sykdom for kunne
hjelpe pasienten. Dette krever ofte innsats fra ulike profesjoner, som bør
arbeide ut fra et felles mål som å hjelpe pasienten i en ny og vanskelig
livssituasjon. (Reitan, 2008;47) Det å bli rammet av kreft og gå gjennom
behandling er ofte forbundet med trusler om tap på flere området. Det kan være
tap av livet, kroppens integritet, sosiale roller, aktivitet, selvoppfattelse,
følelsesmessig likevekt eller det å tilpasse seg ny livssituasjon. (Reitan,
2008;48) Det som ofte skremmer mest ved å få en kreftdiagnose, er tanker om
spredning og død. Pasienten får mange spørsmål og kjenner på følelser som
angst, redsel, fortvilelse, sorg, sinne og aggresjon. Man kan ha behov for å
uttrykke sine tanker, selv om de kan være vanskelige å verbalisere. Evnen til å
fortelle om sine tanker varierer fra person til person. For noen er dette
naturlig å snakke åpent om, mens andre er tilbakeholde og holder det for seg
selv. En skal vise respekt for det som er naturlig for pasienten. (Reitan,
2008; 53) Ved alvorlig grad av sykdom kan ensomhetsfølelsen oppleves som stor.
Pasienten kan føle seg verdiløs og forlatt. Ved innleggelse i sykehus blir
pasienten adskilt fra sine nærmeste, dette kan være med å forsterke
ensomhetsfølelsen. Det kan være tungt å gå gjennom prøver, undersøkelser og
behandlinger på egenhånd. (Reitan, 2008;53) Sykdom, behandling og konsekvenser
av dette forutsetter ofte behov for læring og mestring. Man utfordres til å
lære nye sosiale roller, nye måter å håndtere hverdagsliv, tolke kroppens
signaler i tillegg til kunnskap om sykdommen og behandlingen. (Reitan,
2008;53-54) Pasientens evne til å ta imot informasjon kan være begrenset. Som
sykepleier bør en være utvelgende når det kommer til dette, tenke over hva
pasienten har behov for å vite. Formidlingen bygger primært på at det
eksisterer en relasjon mellom den kreftsyke og den som formidler informasjon og
at de kan kommunisere med hverandre. Informasjonen må tilpasses situasjonen,
være enkel og konkret. Den bør formidles på en måte som pasienten forstår og
som er så nær dens egen opplevelsesverden som mulig. (Reitan, 2008; 53)

\section{Relasjonsbygging}

Relasjonen mellom pasient og sykepleier er vesentlig i sykepleiepraksis. Som to
individer med sine svakheter og styrker møtes pasienten og sykepleier.
(Gr.Kunn.i klinisk spl s 899)For å bygge og utvikle en menneske til
menneskerelasjon har sykepleierens egne holdninger, forventninger og åpenhet
for informasjon betydning ( Brataas, 2011 s89). Sykepleier må forholde seg til
pasientens erfaringer som er komplekst nett av meninger, uttrykk og følelser.
(Gr.Kunn.i klinisk spl s 899)For at pasientens skal oppleve psykososial støtte
og at sykepleien er god, er forholdet mellom sykepleier og pasient vesentlig. I
den palliative fasen står gjerne psykososiale behov sentralt. (Gr.Kunn.i
klinisk spl.s 900). For at sykepleier skal kunne ivareta pasientens
psykososiale behov må utgangspunktet være den enkelte unike pasient. Sykepleier
må ta seg tid til å lytte og bli kjent med pasienten for å klare å avdekke hans
behov akkurat nå (Gr.Kunn.i klinisk spl.s 901) Travelbee sier i sin definisjon
av sykepleie at “sykepleie er en mellommenneskelig prosess hvor den
profesjonelle sykepleier hjelper et individ, en familie eller et samfunn med å
forebygge eller mestre erfaringer med sykdom og lidelser, og om nødvendig å
finne mening i disse erfaringene” (Travelbee, 2007, s. 29). At det er en
“mellommenneskelig prosess” forklarer Travelbee er fordi det alltid dreier seg
om mennesker, enten direkte eller indirekte. En sykepleier må forholde seg til
mange mennesker enten det er de syke eller friske, pårørende, besøkende,
personale eller andre helsearbeidere.  At det er en “prosess” understreker det
at sykepleie er noe dynamisk. Det kan være en hendelse eller erfaring hvor
sykepleier har en relasjon til et annet individ som har behov for sykepleierens
hjelp. En slik situasjon vil alltid være i bevegelse, under utvikling eller
tilblivelse (Travelbee, 2007, s. 30).  Humor kan være kontaktskapende når det
blir brukt for å bygge relasjoner (Tyrdal 1, s. 192).  Arnold \&{} Boggs (2003)
sier at humor kan bidra til økt nærhet og kontakt. Det kan styrke bånd mellom
sykepleier og pasient eller pårørende, dersom trygg og god kontakt allerede er
etablert. Den felles gleden ved en spøk, kan styrke pasient og sykepleier
forholdet (referert i Eide \&{} Eide, 2007 s. 245-246/247). Det å gi pasienter
dårlige nyheter kan gi pasienten psykisk smerte. I følge Kari Martinsen (1991)
har omsorg å gjøre med relasjoner og moral som yter seg i praktisk handling.
For å kunne hjelpe pasienten, må sykepleier først og fremst kunne sette seg inn
i pasientens situasjon. (Reitan, 2008; 80-81) Humor kan og brukes på en negativ
måte, til å latterliggjøre eller kritisere andre. Dette må ikke forekomme i
relasjoner til pasienter, pårørende eller kolleger. Spøk og fleip kan ha et
skjult, negativt budskap. (Eide \&{} Eide, 2007. s. 246) Sykepleiersens holdning
og væremåte har betydning for om en situasjon oppleves god eller dårlig av
pasienten. Sykepleierenes utfordring blir å møte pasienten på en god måte. Der
sykepleiers moral og etikk viser seg gjennom væremåte og kroppsspråk, i tillegg
til det som blir sagt og gjort. (Brinchmann, 2008 s.127) Det kan være vanskelig
å hjelpe et menneske å mestre sykdom, og vanskeligere er det dersom
fremtidsutsiktene er dårlige. Gode relasjoner og kunnskap om hva mennesker
trenger for å finne en mening når livet er vanskelig er viktig. Joyce Travelbee
sier at det er først når er godt samarbeid mellom sykepleier og pasient er
oppnådd at man kan hjelpe den syke å finne håp og mening, og til å mestre
sykdom og lidelse.  Videre sier hun at en sykepleier må strebe etter å få til
en forandring, slik at pasientens helse kan opprettholdes på best mulig vis
(Travelbee 2007). En sykepleier må derfor være i stand til å identifisere og
skape forandring på en målrettet, innsiktsfull og omtenksom måte (Travelbee,
2007, s 30). For å få til et godt samarbeid er det i følge Travelbee
sykepleierens mål å skape et menneske-til menneske-forhold mellom sykepleier og
pasienten. (Travelbee, 2007. s 41)

\section{Kommunikasjon}

Kommunikasjon blir brukt for å komme i kontakt med andre mennesker, utveksle
tanker og følelser. Vanligvis kommuniserer vi både verbalt og non-verbalt. Den
non-verbale kommunikasjonen inkluderer kroppsholdning, kroppsspråk og
ansiktsuttrykk med mer. Uten den non-verbale kommunikasjonen kan den verbale
kommunikasjonen være vanskelig å forstå. (Reitan, 2008. s.67) Kommunikasjon er
mer enn formidling av informasjon. God kommunikasjon og samhandling med
kreftpasienten og dens pårørende er grunnleggende for utøvelse av god
sykepleie. Travelbee ser på kommunikasjon som en dynamisk prosess og et
instrument i sykepleiesituasjoner. Det er først og fremst et middel som brukes
for å bli kjent med pasienten, til å forstå og møte pasientens behov for hjelp
til å mestre sykdom, lidelse og ensomhet. I følge Travelbee kan ferdigheter som
god kommunikasjon læres dersom sykepleieren har en intellektuell
tilnærmingsmåte til problemene, sammen med at hun bruker seg selv terapeutisk
(Reitan, 2008 s.65). Selv om sykepleier skal ha vilje og engasjement til å møte
pasientens behov og problemer, må det være rom for at sykepleieren kan erkjenne
sin egen usikkerhet og ubehag ved å gå inn i vanskelige samtaler (Reitan, 2008
s.81).  Mennesket er et humoristisk og lattermildt vesen. En studie på eldre,
nyopererte pasienter viser at pasientene synes at sykepleierne brukte for lite
humor. Når humor ble brukt, var det pasientene selv som tok initiativ (Borge
\&{}
Kristoffersen 2002). Humor kan være viktig for pasientens befinnende, men også
for trivselen på arbeidsplassen som en mestringstrategi for både pleier og
pasient. Humor kan brukes til å utfordre andre og redusere spenning (Eide \&{}
Eide, 2007. s. 242-243).  Det er ingen regler å forholde seg til når det kommer
til humor. Pasienter er forskjellige og reagerer ulikt. Man må alltid se an
personen og situasjonen. Dette medfører at humor kanskje ikke egner seg før man
kjenner pasienten forholdsvis godt (Arnold \&{} Boggs, 2003) (Eide og Eide, 2007.
s.247). Humor er en individuell og personlig sak, det må brukes en viss
varsomhet rundt dette (Olsson mfl.2001) \cite[s.~394]{bjork2011}.

Vennligsinnet humor kan være god medisin. Det kan være like viktig for å
oppleve livsglede som omsorg og fellesskap (Bøhn, 2000. s.55). Humor og latter
har alltid vært en del av menneskets sosiale omgangsform. Det er en verdifull
egenskap, når den brukes med forstand. Det hevdes at mennesker med lyst sinn
både er friskere og lever lenger enn de med et dystert sinn (Bøhn, 2000 s.55).
Humor kan bidra til å løsne på stemningen og redusere trykket av vanskelige
følelser og tanker, og det kan styrke kontakten mellom sykepleier og pasient.
Men på en annen side kan humor oppfattes som kunstig munterhet som virker
fremmedgjørende. Det kan være en metode for å unngå seriøse samtaler (MEchanic
1991), både fra sykepleier og pasientens side. (Eide og Eide,2007 s. 247) Som
sykepleier kan det ofte være lurt å lytte til pasienten, la dem bruke humor
rundt sin egen situasjon, men være tilbakeholden med egne humoristiske
bemerkninger, særlig før enn kjenner pasienten godt. Humor, latter, lykke og
glede kan observeres hos mennesker i den ytterste krise. Helsearbeidere bør
søke å stimulere disse positive menneskelige uttrykkene gjennom sitt arbeid.  (
Wist, 2002. s. 169) Det er ikke alle former for humor som er akseptable i
profesjonelle relasjoner, vitsing, fleip og moro er ikke alltid på sin plass.
Fleip og humor brukes ofte til å dekke over indre problemer, usikkerheter og
spenninger. Noen bruker humor som en form for kontroll. Galgenhumor i forhold
til sin egen situasjon, kan være en god måte å fjerne seg fra sorg og smerte
for en periode. Det kan gi en opplevelse av fellesskap, om man ler sammen med
noen som har felles skjebne. Ikke alle mennesker reagerer positivt på humor,
men for mange kan det være en hensiktsmessig og virkningsfull
kommunikasjonsstrategi. (Burnard 1992, Arnold og Boggs 2003) (Eide og Eide,
2007 s.244) Humor har en plass i samvirket mellom pasient og sykepleier, men
humoren skal ikke være en brodd mot pasienten. Det må være humor som kan deles,
man skal le sammen ikke av. ( Wist, 2002 s. 166) Blant personalet kan trykket
og stressnivået ofte være høyt, da trenger man av og til å få lette litt på
trykket sammen med kolleger gjennom fleip og humor. Det er viktig at den type
humor som brukes, ofte galgenhumor\index{galgenhumor}, blir holdt internt i
personalet. For pasienter med store lidelser, vill denne type humor kunne
oppleves som krenkende. Enkelte vil kunne spekulere i om latteren har noe med
dem å gjøre.  (Eide og Eide, 2007 s.245) Det er flere former av humor som kan
oppfattes morsomt for noen, mens andre kan ta det som en krenkelse. Slike
former kan være sarkasme, ironi og fleip. Det er viktig å være forsiktig med
bruk av slike. Man må være sikker på at pasienten setter pris på denne type
humor, før man eventuelt bruker det. (Eide og Eide, 2007 s.246-247)

\subsection{Mestring}

Humor kan være en hensiktsmessig mestringstrategi (Mechanic 1991), det kan
fungere som en sikkerhetsventil når undertrykket blir for stort.  Man kan
aktivt bruke humor for å mestre den situasjonen man har kommet i, og det bør
ikke brukes til å fornekte situasjonen, men det kan brukes som et emosjonelt
sverd i en tøff hverdag. Man observerer ofte at pasienter som spøker med
alvoret, reduserer spenningen (Wist, 2002 s 164-165).

Ved alvorlig sykdom blir man fokusert på egne problemer og lidelser, dette
fører til at evnen til å forholde seg åpent til andre blir redusert. Humor kan
i tilfeller som dette bidra til å åpne opp, lette på stemningen og skape
kontakt. Det kan også gjøre det lettere å snakke om det som er vanskelig (Eide
og Eide, 2007 s. 244). God humor kan føre inn et element av forsoning med det
tragiske, fortvilende eller absurde i situasjonen pasienten befinner seg i. På
den ene siden letter humoren på trykket. På en annen side gjør den det mulig å
være nær det som er vanskelig, om enn bare for et øyeblikk (Eide og Eide,2007
s. 244-245). I følge Lange mfl (1990), kan humor virke som en befrielse fra
deprimerende alvor, ved at den kan skape en avstand til alvoret som har en
tendens til å overskygge tilværelsen (Referert i Eide og Eide, 2007 s. 245).
Pasienter kan bruke humor for å skape en avstand til seg selv, som kan hjelpe
dem til å gjennomleve den vanskelige situasjonen \cite[s.~394]{bjork2011}.

Humor kan brukes som en form for forsvar i vanskelige situasjoner (Mechanic
1991). Det kan aktualisere følelser og behov som er vanskelige å forholde seg
til, som fortrenges eller undertrykkes. Dette kan være følelser som angst,
aggresjon eller seksuelle følelser. Freud påpekte dette, han mente at vitsens
grunnlag er våre fortrengte drifter og følelser. Humor gir anledning til
utslipp av impulser som vi ellers holder på avstand (Eide og Eide, 2007 s.243).

\chapter{Metode}

Vi har valgt å gjøre en litteraturstudie med kvalitativ metode som
utgangspunkt. En litteraturstudie er en studie hvor innsamlingsdata hentes fra
litteraturen, og hvor man får en strukturert oversikt over et valgt tema
(Friberg 2006). Kvalitativ metode er en forskningsmetode som gir beskrivende
data. Den søker å gi en dypere forståelse og en økt kunnskap om et tema. Ved
hjelp av intervjuer eller observasjon får man innsikt i menneskers personlige
opplevelser og erfaringer (Olsson \&{} Sörensen 2003). Vi har valgt å analysere
fem kvalitative forskningsartikler etter Fribergs modell i “Dags för uppsats”
(2006). Friberg sier at kvalitative studier gir en økt forståelse for hvordan
man kan møte pasientenes behov gjennom å se på pasientens opplevelser,
erfaringer og forventninger (Friberg 2006).  Vi ønsker å se nærmere på hvordan
man kan bruke humor i arbeid med kreftpasienter, hvordan humor blir mottatt, og
vi ønsker å vite noe om hva pasientene opplever som positivt og negativ ved
bruk av humor. Etter vår mening og det vi tidligere har lært, er kvalitativ
forskning den metoden som er best egnet for besvarelse av vår problemstilling.
Gjennom observasjoner og intervju gir metoden gode beskrivelser av pasientenes

\section{Litteratursøk}

Litteratursøket til denne oppgaven ble gjennomført i perioden 26.09.2015 til
03.11.2015. Databasene vi brukte til å søke opp artikler var Cinahl og Oria. Vi
har også brukt tidsskriftet Scandinavian Journal of Caring Sciences’ egen
søkemotor for vitenskaplige artikler. De fleste artiklene fant vi i databasen
Cinahl. Dette er en database som blir mye brukt til søk av vitenskaplige
artikler innen helsefag, og var derfor godt egnet for oss.  I Cinahl valgte vi
å begrense søket til å gjelde fra 2005 til 2015 slik at vi skulle få de senest
publiserte artiklene, og den nyeste kunnskapen. Vi huket av på” full text” og
“peer reviewed” da dette indikerer at artiklene er vitenskaplige, og
kvalitetssikret av andre. Søkeordene som ble brukt var engelske. Dette fordi de
fleste vitenskapelige publikasjoner utgis på dette språket. Grunnen til det er
at forfatteren vil nå ut til flest mulig med sin forskning (Friberg 2006).
Søkeord som Humor, Humour, Health care og Nurs ble brukt. Vi fant fire artikler
her som vi ønsket å undersøke videre.

I Oria huket vi av for \textit{artikler} og brukte søkeordene \textit{humor} og
\textit{relationship}. Her fikk vi 1464 treff. Videre avgrenset vi for årstall
fra tidsrommet 2005-2015 og at artiklene skulle være fagfellevurdert.  Dette
førte til 584 treff. Blant dem fant vi én som vi ønsket å se nærmere på.
Artikkelen var tilgjengelig i fulltekst hos ProQuest Health and Medical
Complete.

Den siste artikkelen vi ønsket å se nærmere på fant vi i Scandinavian Journal
of Caring Sciences. Her ble søkeordene \textit{humour} og \textit{cancer}
brukt. De fleste artiklene her ble forkastet da de ikke oppfylte våre
kriterier, og av 17 treff var bare én av disse relevante for oppgaven vår.

\subsection{Oversikt over analyserte artikler}

Forfattere (år)
	Tilnærming
	Metode
	Deltakere
	Søkemetode
	B,L. Roaldsen, T. Sørlie
(2015)
	Kvalitativ studie
	Intervju i perioden 2010-2011 av første forfatter.
	14 pasienter i alderen 23-83 år.
7 menn og 7 kvinner.
	Søk i tidsskrift
	M,A. Tanay, T. Wiseman, J. Roberts, E. Ream. (2013)
	Kvalitativ studie
	Intervju, observasjon, semi-
strukturerte intervju og uformelle intervju.
	9 sykepleiere og 12 pasienter ble observert.
5 pasienter og 5 sykepleiere ble intervjuet
	Søk i database
	Dean. R,A. Kinsman, D. Gregory. (2005)
	Kvalitativ studie.
	Observasjon av sykepleiere over 6 uker. Intervju med pasienter og pårørende. Semi-
strukturerte intervju med helsepersonell.
	6 sykepleiere ble observert, 11 ble intervjuet. 2 sosial arbeidere, 1 fysioterapeut.
	Søk i database
	M. McCreaddie, S. Payne. (2011)
	Kvalitativ studie.
	Intervju, observasjon, notater, lyd dagbøker.
	4 pasient fokus grupper.
32 deltakere.
	Søk i database
	R, A. Kinsman, J, E. Major. (2007)
	Kvalitativ studie
	Sammenlig-ning av to tidligere studier gjennomført av forfattere.

	Søk i database
	
\section{Analyse}

Analysen av tekstene har vi gjort med utgangspunkt i Febe Fribergs
analysemodell i “Dags för uppsats” (2006). Friberg mener det er viktig å
fokusere på studienes resultat, og lese gjennom artiklene flere ganger slik at
man får ordentlig tak på hva de handler om. På denne måten blir det enklere å
plukke ut hovedfunn og gjøre en sammenligning mellom de ulike artiklene.

Friberg beskriver analysearbeidet som en bevegelse. Når man har valgt ut sine
artikler, deler man artiklenes resultater inn i de kategorien man søker.
Deretter setter man det sammen igjen til et nytt resultat. Her ser man at
arbeidet går fra en helhet, til deler og tilbake til en ny helhet (Friberg
2002).

Etter å ha lest gjennom abstraktet, valgte vi ut de artiklene vi syntes var
mest relevante. Det var vanskelig å finne artikler som omhandlet både kreft og
humor, og som samtidig holdt opp til våre kriterier. Derfor valgte vi å bruke
noen artikler som omhandler palliasjon, da dette kan være en naturlig del av et
kreftforløp. Med hovedfokus på tema og resultat, leste vi gjennom alle
artiklene for å få en oversikt over hva de handlet om, og om de oppfulgte våre
kriterier. Noen av artiklene ble byttet ut underveis, da vi ikke syntes de ga
svar på det vi søkte etter. Deretter finleste vi artiklene for å finne likheter
og ulikheter mellom de forskjellige. Gjennom fargekoder og skjematisk
fremstilling sorterte vi de ulike funnene. På den måten fikk vi en god oversikt
over likheter og ulikheter i artiklene. Ut i fra dette kunne vi dele det videre
opp i hoved- og underkategorier. Dette var et krevende arbeid. Det var mange
interessante resultater, og mye vi kunne tenkt oss hatt med, så det var
utfordrende å holde fokus på det vi søkte etter. På bakgrunn av funnene vi
gjorde, satte vi til slutt sammen to hovedtema med tilhørende syv subtema.

Kommunikasjon

	-         Relasjonsbygging
-         Den vanskelige samtalen
-         Individuelle hensyn
	Mestring

	-         Bruk av humor som
          mestringstrategi.

\chapter{Resultat}

\section{Kommunikasjon}

\subsection{Relasjonsbygging mellom sykepleier og pasient}

Artiklene viser at både sykepleier og pasient ser på humor som et tegn på at
det utvikles et personlig forhold mellom dem. Flere pasienter opplevde at de
hadde et godt forhold til sykepleier dersom de lo sammen. Sykepleierne
oppfattet viktigheten humor hadde for å etablere en god relasjon til
pasientene, og effekten det hadde for å få pasientene til å føle seg mer
avslappet. Ved bruk av humor fikk sykepleiere vise at de også var mennesker, og
dette gav en følelse av tilhørighet for pasienten og de rundt dem. Det virket
som om pasientene fikk større tillit til sykepleierne dersom det ble brukt
humor. På den måten ble det lettere for pasientene å åpne seg, og snakke om de
vanskelige tingene (Tanaya, Wiseman, Roberts \&{} Ream, 2013, s. 1297).

I studien til Tanya et al (2013, s 1298) så det ut til at sykepleiere
foretrakk morsomme pasienter. Det var lettere for pasienten å ta kontakt med en
sykepleier som viste en form for humoristisk sans i forhold til en som ikke
hadde det.  En pasient sier at man ved bruk av humor hadde større sjanse for å
bli likt, og dersom man brukte humor til å spør om noe, svarte sykepleierne
muntrere.  En pasient beskriver det slik: <<Somebody with a sense of humour
asking for a cup of tea, is more likely to get one than somebody demanding a
cup of tea>>.

McCreaddie og Payne (2001) finner i sin studie at humor ikke alltid er
positivt. Noen pasienter kjenner på behovet for å være en “god” pasient, og
prøver dermed å adoptere sykepleieren sin væremåte. Dersom sykepleieren bruker
humor, gjør også pasienten det for å oppnå bedre kontakt, og for å få den
hjelpen de trenger.  De finner også at pasientene bruker humor til å uttrykke
sine følelser og engstelse. Dette kan være risikabelt da sykepleier ikke
oppfatter alvoret fra pasienten.

\subsection{Den vanskelige samtalen}

Sykepleier mente at humor fikk pasientene til å føle seg hjemme og være mer
avslappet. De tok dette som en indikasjon på at pasienten stolte mer på dem
dersom de hadde ledd litt sammen. Dette førte til at pasienter åpnet seg mer,
og ønsket å snakke om de seriøse tingene (Tanay et al 2013, s. 1297) Humor ble
brukt som en beskyttelse av pasientene når man for eksempel gikk tom for ord,
eller for å slippe å vise sårbarhet (Roladsen et al, 2015, s.4). Å sette ord på
erfaringer kan være vanskelig, og det å bruke humoristiske uttrykk kunne være
en indirekte måte å kommunisere forståelig med andre. De brukte humoristiske
metaforer og bilder med varierende intensjoner, som for eksempel å stanse
sensitive temaer uten at samtalepartneren skulle føle seg avvist.  Sykepleierne
sa at de ved å bruke humor på en måte kunne distansere seg selv fra pasienten,
og slippe å måtte gå inn i seriøse diskusjoner med dem (Dean \&{} Major, 2007).  I
Roaldsen et al (2015) reagerer en pasient med latter når hun får diagnosen
brystkreft. Hun sier at dette var den eneste måten hun greide å reagere på,
samtidig som hun følte hun ikke burde le, fordi pleierne ville tro hun var gal.
Etter normalen skulle man i en slik situasjon gråte, ikke le.

Innen palliativ pleie var det ofte humor involvert når de snakket om fortiden.
Mimring om fortiden var spesielt viktig for pasientene da de ble konfrontert
med at de var på deres siste dager. Av og til delte de burder fra fortiden, men
ofte også høydebunker gjennom livet. I slike situasjoner var ofte latter
involvert. Dette var også en mulighet for personalet å få god kontakt med
pasienten og deres pårørende (Dean \&{} Gregory, 2005).

\subsection{Individuelle hensyn ved bruk av humor}

I studien til Dean \&{} Gregory (2005) kommer det fram flere faktorer som er
avgjørende for bruk av humor i sykepleien. I tillegg til etnisitet, kjønn,
stressnivå etc. avhenger bruk av humor på pasientens personlighet og
situasjonen pasienten er i. I tillegg uttrykte noen sykepleiere en utrygghet
ved bruk av humor i jobben. De fryktet at det skulle gå utover deres
profesjonalitet, og de var bekymret for hvordan deres kolleger skulle se på
dem. Dette kan være grunnen for at mange unge sykepleiere var mer seriøse på
jobb i møte med eldre sykepleiere. Når det bare var unge sykepleiere ble det
observert mer humor og tøys. (Tany et al 2013, s. 1298)

Det ble pekt på flere omstendigheter hvor deltakerene mente at humor ikke var
passende. Ved endring i pasientens tilstand kan det være mye frykt, sinne og
sorg. I slike situasjoner ble forsøk på bruk av humor ikke satt pris på av
pasientene. Det kom frem flere situasjoner hvor pleierne forsto at de hadde
gått for langt, og at humoren ikke passet.

I situasjoner hvor pasienten ligger på dødsleiet var det av den oppfatning at
humoren skulle bli overlatt til pasienten og pårørende. Det var her ikke
passende for personalet og komme med kommentarer av humoristisk karakter .  En
sykepleier forteller om rørende øyeblikk rett før døden <<in their last minutes
of life I’v seen humour used there too. It’s a very loving humour, it’s kind og
heart-to-heart humour from a family member to the one who’s dying>> (Dean \&{}
Gregory 2005).

Både pasient og sykepleier mente at det var viktig å ta hensyn til etnisitet.
Det var likevel vanskelig for deltakerene i undersøkelsen å gi noen spesifikke
eksempler på hensyn man måtte ta, utenom at det var et behov for sensitivitet
og forsiktighet rundt bruk av humor. En pasient mente at humor var vanskelig
mellom mennesker fra forskjellige kulturer, på grunn av misforståelser. Språket
gjorde at det oppsto missforståelser som følge av feil uttalelser og at ting
ble tolket feil .  Det var forskjell i hva kvinner og menn foretrakk ved bruk
av humor. Menn hadde en tendens til å bruke humor som et middel for å være åpne
med hverandre, og for å dekke over ubehag. Deres bruk av humor var også mer
preget av seksuelle bemerkninger over personal, noe som gjorde at mange
kvinnelige sykepleiere følte seg utilpass. Noen håndterte dette ved å overse
kommentarene, mens andre vitset det bort med en spøk.

Det var vanskelig for deltakerene å svare på når og hvordan de brukte humor.
Mange svarte at det ikke var noe bevisst, men at det bare oppsto spontant,
eller at de hadde en intuisjon om at det passet. Andre sa at det var vel
overveid, at det var viktig å være forsiktig og at timingen må være rett (Dean
\&{}Gregory, 2005).

\section{Mestring}
\index{mestring}

\subsection{Bruk av humor som mestringstrategi}

Pasienter snakket om humor som forsvant og humor som returnerte. Humor som
forsvant relaterte de til uvisse livssituasjoner med mye fysisk og psykisk
stress, som involverte angst og usikkerhet om sykdommens utvikling, behandling
(enadeguate) informasjon, ventetid og utilstrekkelig kontinuitet i oppfølging.
Når situasjonen var mer avklart, kunne humor ofte komme tilbake men den var
endret. De brukte typisk galgenhumor\index{galgenhumor} og ironi rettet mot
erfaringene av kreft.  (Cancer survivors` experiences of humour while
navigating through challenging landscapes-  a sosio-narrative approach. 3) I
perioder med kaos og fortvilelse, var humor særlig viktig for å håndtere
følelsene. Slik at sykdom ikke tok opp all plass og var overveldene.  (Cancer
survivors` experiences of humour while navigating through challenging
landscapes-  a sosio-narrative approach. 3) Humor hjalp pasientene til å
distansere seg fra negative følelser og tunge tanker.  (Cancer survivors`
experiences of humour while navigating through challenging landscapes-  a
sosio-narrative approach. 5) Som tidligere nevnt tok mange i bruk
galgenhumor\index{galgenhumor}. Dette gjorde de for å dekke over hvor vanskelig
de hadde det og for å beskytte seg selv. (Cancer survivors` experiences of
humour while navigating through challenging landscapes-  a sosio-narrative
approach.  5) i more than trival: strategies for using humor in palliativ care
skriver de at humor kan bli brukt som distraksjon. innen palliativ pleie kan
pasientene bli avhengige av hjelp fra andre til sensitive og personlige
områder. humor kunne her hjelpe til å distrahere pasienten samtidig som det
hjalp pasienten å ivareta sin personlige integritet.  Pasienter sier at deres
mestring blir satt på prøve under en kreftbehandling. De er derfor bevisste på
sine mestringsstrategier, og understreker at humor er viktig for å skape en
levelig situasjon. Humor bidrar til at sykdom og behandling ikke overskygger
hele deres eksistens.  (Cancer survivors` experiences of humour while
navigating through challenging landscapes-  a sosio-narrative approach. 3)
Muligheten til å akseptere og tåle sykdom var knyttet til et humoristisk
livssyn som gav opplevelse av mening, å være deltakende i verden. Dette dannet
en holdning til seg selv og miljøet som omfatter sykdom og livet generelt. Det
betydde å måtte takle motgang og akseptere sykdommen som et faktum. Pasienter
sa at det var viktig i denne sammenheng å finne en meningsfull balanse, der
sykdommen ikke dominerte.  (Cancer survivors` experiences of humour while
navigating through challenging landscapes-  a sosio-narrative approach. 6)
Data viste at pasienter så på humor som hovedsakelig positivt. De refererte til
ord som positiv, glad, godt humør, mer behagelig osv. (a time to weep and a
time to laugh s. 1297). Flere pasienter identifiserer at deres bruk av humor er
påvirket av graden av stress de erfarte. Det interessante her er at det mer
vanlige svaret var at, dess høyere stress de erfarte, til mer humor brukte de.
(More than trival: strategies for using humor in palliative care. 6 )

Humor og latter var viktig både for helsepersonell og pasient i situasjonen de
var i, for å håndtere spenningen og tristheten som ofte dukket opp. Dette
skapte pusterom fra de tunge situasjonene de befant seg i. (more than trival:
strategies for using humor in palliative care. 4) Innen palliativ pleie brukte
sykepleiere ofte humor som distraksjon. Pasienter blir ofte avhengige av hjelp
fra andre til sensitive og personlige ting. I disse situasjonene kan humor
brukes for å bevare pasientens verdighet. (more than trival: strategies for
using humor in palliative care. 9) innen paliativ pleie, hvor døden kan være
nær, hadde følelser en tendens til å være forsterket. humor kan maskere det en
underliggende følelse, dette kan sende signaler som er lette å feiltolke. De
erfarne sykepleiere som jobbet rundt disse pasientene, lærte seg å høre den
meldingen som var gjemt bak humoren.

\chapter{Diskusjon}

\section{Metodediskusjon}

Vi ønsket å se nærmere på bruk av humor som kommunikasjonsmetode og som
mestringstrategi i sykepleien. Vi ønsket å få et innblikk både i hvordan
pasientene opplever bruk av humor, og hvordan sykepleiere bruker humor i sitt
arbeid.  Å gjøre en empirisk undersøkelse ville være altfor tidkrevende for oss
å utføre, og vi har derfor valgt å gjøre en litteraturstudie hvor det er gjort
bruk av kvalitative forskningsartikler.  Kvalitativ metode gir oss et
perspektiv sett innenifra, og gir en dypere forståelse for et fenomen blant
annet gjennom intervjuer og observasjon.  Man får på den måten et bedre
innblikk i menneskers tanker, opplevelser og følelser (Olsson \&{} Sörensen 2003).
Det kan være vanskelig å få et bilde av tanker og følelser gjennom tall og
statistikk, og vi syntes derfor at kvalitativ metode var bedre egnet enn
kvantitativ metode i denne oppgaven.

Litteraturstudier innhenter data fra allerede analysert materiale som
vitenskaplige artikler (Friberg 2006). Metoden har blitt kritisert fordi den
ikke kommer frem med ny forskning, at det er for lite utvalg av materiale og at
den kan bli for subjektiv. Forskerene eller forfatterne leter etter det de
ønsker å finne, slik at andre kanskje relevante funn kan bli oversett (Frigberg
2006).

Det er blir gjort relativt lite studier om temaet humor i sykepleie, og
utvalget av artikler som var relevante for vår problemstilling var dermed
begrenset. Bare én av de fem artiklene vi valgte ut er norsk, mens de
resterende er fra engelskspråklige land. Vi fant svært få undersøkelser som var
gjort i Norge eller Skandinavia forøvrig, men vi tenker at samfunn og kultur er
såpass likt, at artiklene likevel kan overføres til norske forhold. Vi har ikke
sett på studier hentet fra andre deler av verden, og det kan vel derfor tenkes
at denne studien blir rettet mot den vestlige verden. Man kan jo anta at man i
andre deler av verden har kulturforskjeller hvor man har et litt annet syn på
bruk av humor enn oss, noe som igjen ville gitt oss et litt annet resultat.
Selv om det var et begrenset utvalg artikler, besvarer disse likevel vår
problemstilling, og vi ser at resultatene i artiklene er gjenkjennelige fra
egen praksis og erfaringer.  Artiklene er hentet fra tre ulike databaser og er
fra forskjellige land. Likevel ser vi at det er mye de samme temaene som går
igjen, og at både hensikt og resultater samsvarer ganske mye. Dersom vi hadde
utvidet søket og hatt enda flere artikler kan det hende at vi hadde fått et
litt bredere resultat enn det vi har. Da resultatene likevel virket å være
såpass like, valgte vi heller å plukke ut noen få artikler, og fokusere på noen
tema som gikk igjen. Før vi startet søket på artikler, undersøkte vi litt rundt
i litteraturen generelt for å se hva vi fant rundt temaene “humor” og “kreft”.
På bakgrunn av dette dannet vi oss et bilde av hva vi ønsket å se nærmere på,
og det gav oss en pekepinn på hvilke søkeord vi skulle bruke. Underveis i
analysearbeidet endret fokus seg noe, og tema som “humor og fysiologisk
betydning” ble for eksempel valgt bort, da vi heller ønsket å rett fokus mot
kommunikasjon og mestring.  Tre av artiklene er av nyere dato, mens de to andre
er noen år eldre. Vi valgte likvel å ta disse med da vi ser at de er like
dagsaktuelle som de andre.

I fire av artiklene er datainnsamling hentet inn via intervju og observasjon,
mens den siste artikkelen er basert på funn fra sammenligning av to
vitenskaplige artikler. Det er gjort både semistrukturerte og uformelle
intervju.  I den ene  artikkelen er det kun hentet inn data via intervju, ikke
observasjon. I samtlige artikler hvor data er hentet inn via intervju og
observasjon er informantene helsearbeidere og pasienter. I den ene studien er
det også pårørende som informanter, men disse har vi imidlertid ikke valgt å
legge vekt på da vår fokus ligger på pasient og sykepleier.

I kvalitativ metode skal man kunne gå i dybden og få en nærhet til det man
ønsker å undersøke. Det er da en forutsetning at man har et mindre antall
informanter (Olsson \&{} Sörensen 2003.) I artiklene vi har valgt gjenspeiles
dette ved at  det  hovedsaklig er små grupper på mellom 20 - 30 informanter.
Ved et lite antall informanter er det enklere for forskeren å få en relasjon
til vedkommende, og på den måten få et bedre innblikk fra informantens
perspektiv. Informantene i studiene er

Hvordan plukket ut informantene, kriterier for å delta??  Forfatternes metode
for datainnsamling ??  Valgt bor å se på bruk av humor som mestringstrategi for
seg selv.

\section{Resultatdiskusjon}

\todo{Hva skal diskuteres}

\begin{itemize}
\item Må ikke være en oppramsing av resultatene --- begrunna resultatene, Viser
  de ulike artiklene ulike resultater, begrunne begge er en styrke.

\item Hva er vårt kunnskapsbidrag

\item På hvilken måte bidrar resultatene å forbedre dette i praksis

\item Gi konkrete eksempler på hvordan omsorgsarbeidet kan påvirkes av våre
  resultater

\item Koble eksamensarbeidet til omsorgsvitenskaplig utgangspunkter

\item Reflekter over om resultatet kan overføres til samfunnets mål og
  økonomiske rammer ut i fra vår erfaring….

\item Viktig å være kritisk og kunne distansere seg i diskusjonen
\end{itemize}

\todo{Utgangspunkt for diskusjon}

\subsection{Kommunikasjon}

Som tidligere nevnt er humor et hjelpemiddel som kan tas i bruk i samhandling
mellom pasient og sykepleier. Vi ønsker å drøfte de funnene vi har fra
artiklene og teorien opp mot vår problemstilling. Vi ha delt opp drøftingen i
to deler, kommunikasjon og mestring. under kommunikasjon tar vi opp
relasjonsbygging, den vanskelige samtale og forhold man må ta hensyn til. I
delen om mestring, ser vi på mestring fra pasientens side.

I teorien kan vi se at relasjonen mellom sykepleier og pasient er vesentlig for
god sykepleiepraksis. Kreftpasienter går gjennom en vanskelig tid og har behov
for støtte. Som sykepleier må man forholde seg til pasientens erfaringer, som
er en kompleks sammensetning av meninger, uttrykk og følelser. Man må som
sykepleier ta utgangspunkt i den enkelte pasient og ivareta dens psykososiale
behov (Gr. kunn. i klinisk spl. s. 901). I Tyrdal (2002; 192) ser vi at humor
kan være kontaktskapende når det blir brukt for å bygge relasjoner. I Eide og
Eide (2007; 245-246) (McGee 1985, Arnold og Boggs 2003) Sier de at humor kan
bidra til økt kontakt og nærhet. I samme teori fant vi også at humor kan styrke
båndet mellom sykepleier og pasient, dersom god og trygg kontakt allerede er
etablert (2007; 245-246). Dette samsvarer med det vi fant i artiklene, der
kommer det frem at både sykepleier og pasient ser på humor som et tegn på at
det utvikles et personlig forhold mellom dem. Artiklene viser at humor var
svært viktig for å unngå å miste meningsfulle forhold. For flere av pasientene
i disse studiene opplevdes det godt, dersom de lo sammen med sykepleier.  Innen
kreftomsorg kommer det ofte situasjoner der man må gå inn i vanskelige
samtaler, dette kan være krevende både for pasient og sykepleier. Det kan være
samtaler rundt diagnoser, forventninger, frykt og videre. I teori finner vi at
det er grunnleggende for god utøvelse av sykepleie, at man har god
kommunikasjon med pasienten. Det er et middel som brukes til å forstå og møte
pasientens behov for hjelp til å mestre lidelse, sykdom og ensomhet (Reitan,
A.M. \&{} T.Kr. Schjølberg (red.) 2004; 65). I artiklene kom det frem at
sykepleiere mente at humor fikk pasientene til å føle seg mer avslappet og
hjemme. Dette tok sykepleierne som en indikasjon på at pasientene stolte på
dem. Både i teori og praksis fant vi at humoren også kunne brukes for å unngå
vanskelige samtaler. Eide og Eide (2007; 247) (Mechanic 1991) skriver at humor
kan oppleves på en måte som virker fremmedgjørende. Det kan være en metode for
å unngå vanskelige samtaler, både fra pasient og sykepleiers side. Artiklene
sier at det kunne brukes som et skjold, for når man er tom for ord eller for å
slippe å vise sårbarhet. Det å sette ord på vanskelige erfaringer kan være
vanskelig. Humor kunne også være positivt i forhold til egne følelser. I Tyrdal
(2002; 169) viser de til at humor, latter, lykke og glede kan observeres hos de
mennesker som er i den dypeste krise. I artiklene fant vi at pasientene brukte
humoristiske uttrykk for å indirekte kommunisere forståelig med andre. Det
bidro til å løsne på stemningen og redusere trykket ved vanskelige tanker.

Det er ingen regler å forholde seg til når det kommer til humor, pasienter er
ulike og reagerer ulikt. Humor og latter har alltid vært en del av menneskets
sosiale omfang. Det er en verdifull egenskap, men den må brukes med forstand (
Bøhn, M. (red), 2000; 55). Det er viktig at an tar hensyn til den andre
personen og situasjonen man befinner seg i. I Eide og Eide (2007; 247) (Arnold
og Boggs 2003) ser vi at det samsvarer med annen teori, de skriver at det er
flere situasjoner hvor det ikke egner seg å bruke humor, eller det må brukes
med varsomhet. I artiklene har vi funnet flere eksempler på tidspunkt hvor bruk
av humor ikke er optimalt. Dette viser at det er enighet om temaet her også.  I
artiklene tar de opp ulike situasjoner, som ved endring av pasientens tilstand.
Her kan humoren ofte bli misforstått. Dette er en tid som ofte er preget av
frykt, angst, sinne og sorg. En annen ting som er viktig å ta hensyn til er
etnisitet, her kan det lett oppstå misforståelser. På grunn av
kulturforskjeller, vanskeligheter med uttalelse og feiltolkede betydninger.
Andre hensyn som er nevnt i artiklene er blant annet kjønn, alder og tid på
døgnet. Eide og Eide (2007; 247) (Arnold og Boggs 2003) skriver at humor
kanskje ikke egner seg så godt før man kjenner pasienten forholdsvis godt, da
alle mennesker er ulike og reagerer ulikt. I (Grunnleggende sykepleie 2 s. 294)
(Olsson mfl. 2001) ser vi at humor er en individuell og personlig sak, så det
må brukes en viss varsomhet rundt det.  Fra artiklene kan vi se at flere
pasienter uttrykte at  humor var særlig viktig for å unngå å miste meningsfulle
forhold. Det så ut som om sykepleierne foretrakk morsomme pasienter.
Sykepleierens holdning og væremåte har betydning for om en situasjon oppleves
god eller dårlig av pasienten. Sykepleierens utfordring blir å møte pasienten
på en god måte. Gjennom væremåte, kroppsspråk og det som blir sagt vises
sykepleierens etikk og moral (etikk i sykepleie s. 127) et virkemiddel for å
skape gode relasjoner er ifølge Travelbee (2007) kommunikasjon. Hvor tanker og
meninger kommer frem gjennom språk, tegn, ansiktsutrykk, holdninger og atferd.

\subsection{Mestring}

Når man får en kreftdiagnose, står man overfor flere store utfordringer. Det
innebærer ofte angst, uro og bekymringer for egen livssituasjon. Man kan
oppleve traumatiske kriser, hvor ens tidligere erfaringer og reaksjoner ikke
strekker til for å mestre og forstå situasjonen man er i. Vi har forskjellig
grad i forhold til mestring i kriser (Reitan A.M. \&{} T.Kr. Schjølberg, 2004;
48).Vi takler generelt sykdom på forskjellige måter, sykdom kan gi ulike
reaksjoner, som avhenger av fase i livet, sosialt nettverk og tidligere
erfaringer (Reitan A.M. \&{} T.Kr. Schjølberg, 2004; 39-40) Det er derfor viktig
at vi som sykepleiere vet noe om kriser og reaksjoner. Dette påvirker
livskvaliteten. Fremtiden er usikker og det kan medføre mye lidelse for
pasienten. I Reitan A.M. \&{} T.Kr. Schjølberg (2004; 40-42) kan vi se at ens
selvbilde, troen på at man kan tilpasse seg eventuelle begrensninger i livet og
evnen til å løse problemer er sentrale problemer. Om man prøver å sette
lidelsen i et annet kontekst, blir forventningene mer i samsvar med det enn
erfarer. Dette gir pasienten mulighet til opplevelse av et meningsfullt liv.
Videre i teorien kan vi se at selve kreftbehandlingen er en trussel mot
pasientens eget selvbilde og egenverd. Det kan komme av at man føler seg svak,
trett og slapp, samt har en endret kropp eller endrede kroppsfunksjoner. Det
kan være vanskelig å hjelpe et menneske til å mestre sykdom og vanskeligere
dersom fremtidsutsiktene er dårlige. Gode relasjoner og kunnskap om hva et
menneske trenger for å finne mening med livet når det er vanskelig, er viktig
for en sykepleier. Joyce Travelbee (2007) sier det er først når man har oppnådd
et godt samarbeid mellom sykepleier og pasient, at man kan hjelpe den syke til
å finne håp og mening til å mestre sykdom og lidelse.  I Åmås. K. O (2007; 8-9)
skriver de at kreft hos de fleste assosierer sterke assosiasjoner til døden.
Flere og flere overlever kreft, men det skremmer oss fremdeles. Da det medfører
at man må lære seg å leve med sin kreftsykdom. De må lære seg å mestre
hverdagslivet. Under kreftbehandling blir pasientens mestring satt på prøve. De
får gjerne nye roller i sitt sosiale liv, utfordringer knyttet til endret kropp
eller kroppsfunksjoner. Fra artiklene fant vi at flere av pasientene var
bevisste på sine mestringstrategier og de understrekte at humor er viktig for å
skape en levelig situasjon. Det bidrar til at sykdommen ikke overskygger hele
deres eksistens.  Vi har tidligere nevnt at flere av kreftpasientene ikke
ønsket å ha assosiasjoner til kreft, da de følte at de “ble sin kreft”.
Muligheten til å akseptere og tåle sykdom var knyttet til et humoristisk
livssyn som gav opplevelse av mening og å være deltakende i verden. Dette
dannet en holdning til seg selv og miljøet som omfatter sykdom og livet
generelt. det betydde at de måtte takle motgang og akseptere sykdommen som et
faktum. I Eide og Eide (2007) (Mechaic 1991) ser vi at humor kan være en
hensiktsmessig mestringsstrategi. Noe Tyrdal (2002) understreker, og sier
videre at det er noe man aktivt kan bruke for å mestre den situasjonen man
befinner seg i. Våre funn fra artiklene viser at pasienter hovedsakelig ser på
humor som noe positivt, de refererte til ord som positiv, glad, godt humør og
mer behagelig i denne sammenheng. De forteller at det kan brukes som et
emosjonelt skjold for å takle en tøff hverdag, men her er det viktig at man er
bevisst på at det ikke brukes til å fornekte situasjonen. Videre fant vi at
humor var viktig både for sykepleiere og pasienter i de situasjonene de befant
seg i, det var viktig for å kunne håndtere spenningen og tristheten som oppsto.
Det skapte et pusterom fra det vonde. Tyrdal (2007; 164-165) skriver om
observasjoner av pasienter som spøker med alvoret rundt sine situasjoner og at
dette reduserer spenningen. I Eide \&{} Eide (2007; 243) Ser vi at god humor kan
føre et element av forsoning med det tragiske, fortvilende eller absurde i
situasjonen som er vanskelig å forholde seg til, som fortrenges eller
undertrykkes.

\chapter{Anvendelse av resultatene i praksis / avslutning}

Humoren hører hjemme over alt. Både i hjemmet og på arbeidsplassen. Humor har
alltid blitt omtalt som livets medisin, og en viktig helsefaktor. Det har blitt
gjort noen få vitenskaplige undersøkelse vedrørende humorens effekt i arbeidet
med syke, og det man ser er at bruk av humor berører flere aspekter ved et
menneske. Blant dem fysiologiske, psykologiske og sosiokulturelle… Bruk av
humor er en kunst, som man lærer.
