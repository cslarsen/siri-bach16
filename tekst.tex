\chapter{Innledning}

Kreft er en sykdom som rammer mange mennesker. Det er en alvorlig og
livstruende sykdom som innebærer mange utfordringer og følelser som angst, uro
og bekymringer for egen livssituasjon. En slik omveltende forandring vil kunne
påvirke livskvaliteten og livet generelt. Flere opplever at de stenger omverden
ute, eller at andre trekker seg unna når de har fått en kreftdiagnose (Rustøen,
2008: 39-47).

I dag lever man lenger med kreft enn hva man gjorde tidligere. Dette medfører
at pasientene har andre behov og at helsepersonell har flere oppgaver. Det
handler ikke bare om pasientens fysiske helse, men også hvordan man kan øke
pasientens livskvalitet ved å gjøre livet til den enkelte bedre (Rustøen, 2008:
39).

Kreft er ikke en sykdom som forbindes med noe komisk. Det forventes at man skal
gråte, ikke le. Selv om man får en kreftdiagnose, betyr det nødvendigvis ikke
at man mister sin humoristiske sans. Livssituasjonen og behovene kan endres,
men man er fremdeles det samme mennesket. Bruk av humor i kreftomsorgen kan ha
viktig betydning for pasienten, det kan hjelpe til å bevare følelsen av å være
et levende menneske (Tyrdal 2002: 161-163).

\section{Bakgrunn for valg av tema}

Etter hjerte- og karsykdommer er kreft den vanligste dødsårsaken i Norge. Hvert
år dør i overkant av 10 000 mennesker av kreft. I 2014 fikk 31 651 nordmenn en
kreftdiagnose, og 10 971 døde. Risikoen for å få kreft øker med alderen, men
rammer alle aldersgrupper (Grasdal, 2016). Stavanger Aftenblad skriver at
Rogaland er et av de fylkene i Norge som er hardest rammet av kreft. Man regner
med at så mange som én av tre vil få en kreftdiagnose før fylte 75 år.
Hovedgrunnen til dette er at vi i dag lever lenger (Haugan, 2015).

Som sykepleierstudenter og fra tidligere erfaring har vi merket oss at humor
kan ha stor positiv effekt på samarbeidet med pasienter. Vi vet at humor kan
være kontaktskapende når den brukes til å bygge relasjoner. Vi opplever at det
er med på å åpne opp for et godt samarbeid, og å skape en god relasjon mellom
sykepleier og pasient. Vi har også erfart at man gjerne blir litt mer
tilbakeholden til bruk av humor ved alvorlig sykdom, og gjerne ekstra ved den
siste fasen av livet. Vi ønsket derfor å se litt nærmere på hvordan humor kan
brukes i kreftomsorgen, og hvilken effekt det har i møte med pasienter som
lider av kreftsykdom.

\section{Problemformulering}

Kreft er en alvorlig og livstruende sykdom. Mange sykepleiere vegrer seg derfor
mot å bruke humor i samhandling med kreftpasienter. Flere er usikre på humorens
virkning på pasienter, eller er “redde” for at humoren kan bli misforstått og
få motsatt effekt. Vi ønsker å finne ut hvordan humor blir mottatt av pasienter
med kreft. Det er viktig at vi som sykepleiere vet hvordan vi skal bruke humor
i kommunikasjon med pasienter, og hvilken virkning humoren kan ha i samhandling
med kreftsyke pasienter. Samt at vi vet noe om fordeler og ulemper ved bruk av
humor.

Det er lite litteratur og forskning på bruk av humor i sykepleie, så det er
vanskelig å vite noe om hvordan det fungerer. Man kan oppleve at det blir tatt
i bruk på feil måte, eller at man er redd for å bruke humor. For å vite hvordan
man kan bruke humor, må man vite noe om hva pasienter opplever som positivt og
negativt ved bruk av humor.

\section{Hensikt}

Oppgavens faglige hensikt er først og fremst å se på hvordan humor blir
praktisert i sykepleien. Vi ønsker å finne ut hvordan humor kan brukes i
relasjon mellom pasient og sykepleier. I tillegg ønsker vi å ha særlig fokus på
humor i kommunikasjon og som mestringsstrategi.

\section{Begrepsavklaring}

\todo{Formatering}
Humor:
Vi har alle ulike oppfatninger om hva humor er, det er derfor vanskelig å gi en presis definisjon. Vi har sett på ulike definisjoner og fant blant annet en som beskriver det slik: “Den mentale evne til å oppdage, uttrykke eller sette pris på elementer av latter eller absurd inkongruens (motstridene) i idéer, situasjoner, hendelser eller handlinger” Tyrdal, 2002: 15). I vår oppgave har vi valgt å forenkle dette ved å si at humor er evnen til å oppfatte noe som er morsomt.

\section{Avgrensning av oppgaven}

Vi har valgt å ha fokus på sykepleier, pasient, og samhandlingen dem imellom.
Vi vet at pårørende er en viktig ressurs for den kreftrammede og for oss som
helsepersonell, allikevel har vi valgt å ikke legge vekt på dette i vår
oppgave. Mye av teorien har vi funnet fra Reitan \&{} Schjøldberg (2008), Eide
\&
Eide (2008), Travelbee (2001) og Tyrdal (2002) sine bøker. Dette er bøker som
gir oss sykepleier- og pasientperspektiv. Pasientene i oppgaven er voksne
personer med en kreftsykdom. Selv om oppgaven omhandler kreftpasienter, er det
ikke sykdommen som er vårt hovedfokus.

\chapter{Teori}

Vi har delt teorien opp på følgende måte: I første del skriver vi litt generelt
om kreft, krise og livskvalitet. Videre ser vi på kommunikasjon og relasjon med
humor som virkemiddel. Her har vi fokus på forholdet mellom pasient og
sykepleier. Avslutningsvis ser vi på hvordan humor blir brukt som en
mestringsstrategi ut i fra et pasientperspektiv. På grunn av Travelbees’ fokus
på det mellommenneskelige i utøvelsen av sykepleie, har vi har valgt å ha med
hennes perspektiv på de ulike temaene.

\section{Kreft}

Kreft har blitt en folkesykdom som rammer en stadig større del av den norske
befolkningen.  I dag er det flere som lever lenge med sin kreftsykdom. Hos
mange utløser ordet kreft sterke assosiasjoner til døden. Det er en skremmende
sykdom som kommer innenfra ens egen kropp. Flere av de som rammes av kreft kan
oppleve at livet blir forandret eller forkortet. Flere kreftsyke ønsker ikke å
bli assosiert med sykdommen, da de føler at de i offentligheten “blir” sin
kreft (Åmås, 2007: 8-10).

\subsection{Krise}

Et menneske med kreft står overfor store utfordringer, som kan innebære angst,
uro og bekymringer for egen livssituasjon. De kan oppleve en krise. Kjennetegn
på en krise, er at personens tidligere erfaringer og reaksjoner ikke er nok til
å forstå og mestre situasjonen de befinner seg i. Kriser har forskjellige
styrker og intensitet, og man opplever ulike grader av problemløsning.
Helsepersonell må kjenne til psykiske reaksjoner ved livstruende sykdom for å
kunne hjelpe pasientene. Å bli kreftsyk og det å gå gjennom behandling er ofte
forbundet med trusler om tap på flere områder. Det kan være tap av livet,
kroppens integritet, sosiale roller, aktivitet, selvoppfattelse, følelsesmessig
likevekt eller det å tilpasse seg en ny livssituasjon. Pasientene kan ha behov
for å uttrykke sine tanker, selv om det kan være vanskelige å verbalisere
(Reitan, 2008: 48-53).

\subsection{Livskvalitet}

Livskvalitet er et viktig begrep innen kreftomsorgen, da kreft er en sykdom som
rammer hele mennesket. Vi takler det å bli syk på ulike måter. Samme sykdom kan
gi ulike reaksjoner hos forskjellige mennesker. Dette kan avhenge av hvilke
fase i livet man befinner seg i, det sosiale nettverket rundt en og tidligere
erfaringer (Rustøen, 2008: 39).  Livskvalitet rommer det fysiske, psykiske og
sosiale aspekt ved livet (Rustøen, 2008: 39). Den norske psykologen Siri Næss
har utarbeidet en definisjon på livskvalitet. Hun hevder at: “...et menneske
har det godt og har god livskvalitet dersom man er aktiv, har samhørighet med
andre, har god selvfølelse og en grunnstemning av glede” (sitert i Rustøen,
2008: 39). Noe som er viktig for livskvaliteten er håpet, og da særlig håpet
knyttet til fremtiden. (Rustøen, 2008: 40).  En av grunnene til at kreftsyke
kan oppleve mye lidelse, er på grunn av usikkerhet rundt fremtiden. Det å
hjelpe pasienten med å bevare håpet er noe sykepleier kan bidra med. Her
spiller atferd og tilstedeværelse en viktig rolle. En viktig faktor for å
bearbeide kriser, er å finne mening i situasjonen (Rustøen, 2008: 40-41).
Travelbee gjør rede for metoder som sykepleiere kan bruke for å hjelpe
pasienten til opplevelse av en meningsfull tilværelse. En av disse metodene går
ut på å hjelpe pasientene å innse at alle mennesker opplever lidelse gjennom
livet. Dette kan hjelpe pasientene til å komme over det første sjokket, og til
å starte med bearbeidelse av krisen (referert i Rustøen, 2008: 42).

\section{Humor i kommunikasjon}

Man bruker kommunikasjon for å komme i kontakt med andre mennesker, utveksle
tanker og følelser. Vi kommuniserer vanligvis både verbalt og non-verbalt.
Non-verbalt gjør det lettere å forstå den verbale kommunikasjonen, og kan for
eksempel være ens kroppsholdning, kroppsspråk og ansiktsuttrykk (Reitan, 2008:
67).  Kommunikasjon er mer enn bare utveksling av informasjon. Det er
grunnleggende for utøvelse av god sykepleie at man har en god samhandling med
pasienten. Travelbee sier at kommunikasjon er en prosess i stadig bevegelse,
som kan være et hjelpemiddel i sykepleiesituasjoner. Det er først og fremst et
hjelpemiddel som brukes for å bli kjent med pasienten. Det hjelper oss til å
forstå og møte pasientens behov for hjelp til mestring av sykdom, lidelse og
ensomhet. I følge Travelbee er gode ferdigheter innen kommunikasjon noe som kan
læres. Det krever at sykepleier har en intellektuell tilnærmingsmåte til
problemene og bruker seg selv terapeutisk (referert i Reitan, 2008: 65). Som
sykepleier skal man ha vilje og engasjement til å møte pasientens behov og
problemer, men det må også være rom for at man kan erkjenne egen usikkerhet og
ubehag ved å gå inn i vanskelige samtaler (Reitan, 2008: 81).

Når det kommer til humor er det ingen faste regler å forholde seg til.
Pasienter er forskjellige og reagerer på ulike måter. Man må alltid se an
personen man er sammen med, og situasjonen man befinner seg i. I følge Arnold
\&{} Boggs passer det seg ikke alltid med humor før man er blitt godt kjent med
pasienten (referert i Eide \&{} Eide, 2008: 247). Noe som også bekreftes av Bjørk
\&{} Breievne (2011: 294). I følge dem er humor noe individuelt og personlig som
må brukes med forsiktighet.

Vennligsinnet humor kan være god medisin. Det kan være viktig for opplevelsen
av livsglede, omsorg og fellesskap. Humor og latter har alltid vært en del av
menneskets sosiale omgangsform. Dersom det brukes med forstand, er det en
verdifull egenskap. Det hevdes at mennesker med en positiv innstilling er
friskere og lever lenger, i forhold til personer med en negativ innstilling
(Bøhn, 2000: 55). Det positive med humor er at det kan løsne på stemningen,
redusere vanskelige følelser, tanker og styrke kontakten mellom sykepleier og
pasient (Arnold \&{} Boggs i Eide \&{} Eide, 2008: 247). Som sykepleier er det lurt å
lytte til pasienten. La pasienten bruke humor om sin egen situasjon, men være
tilbakeholden med egne humoristiske bemerkninger. Dette gjelder særlig før man
kjenner pasienten godt. Humor, latter, lykke og glede kan observeres hos
mennesker i en stor krise. Sykepleiere bør ha vekt på å stimulere disse
positive uttrykkene i sitt arbeid (Wist, 2002: 169). I følge Arnold \&{} Boggs kan
humor være en nyttig og virkningsfull kommunikasjonsstrategi ( referert i Eide
\&{} Eide, 2008: 244).

Mechanic uttrykker at en negativ side ved humorbruk er at det kan oppfattes som
kunstig. Det kan bli brukt som en metode for å unngå seriøse samtaler fra både
sykepleier og pasient (referert i Eide \&{} Eide, 2008: 247). I profesjonelle
relasjoner er det ikke alle former for humor som er akseptabelt; vitsing, fleip
og moro er ikke alltid på sin plass. Fleip og humor kan brukes av pasienten for
å dekke over indre problemer, usikkerheter og spenninger. Av noen brukes det
som er form for kontroll. Flere pasienter bruker galgenhumor i forhold til
situasjonen de befinner seg i, dette kan hjelpe dem til å distansere seg fra
sorg og smerte i en viss periode. Humor mellom pasient og sykepleier skal være
noe som kan deles, man skal le sammen og ikke av (Wist, 2002: 166).
Stressnivået kan ofte være høyt blant personalet. Da trenger man av og til å få
utløp for sine tanker sammen med kolleger, da gjerne ved bruk av fleip og
humor. Det kan ofte bli brukt galgenhumor i denne sammenheng. Da er det viktig
at denne type humor blir holdt internt blant personalet, da den kan oppleves
nedverdigende for pasientene (Eide \&{} Eide, 2008: 245). Det var litt mange “da”
her. Går det an å endre litt på dette? :) siri Det er flere former for humor
som kan oppleves krenkende. Eksempler på slike kan være sarkasme, ironi og
fleip. Det er viktig å ha en forsiktighet rundt dette og være sikker på at
pasienten setter pris på det før man bruker denne type humor (Eide \&{} Eide,
2008: 246-247).

\section{Humor som relasjonsbygging}

I sykepleiepraksis er relasjonen mellom pasient og sykepleier vesentlig. To
mennesker med sine styrker og svakheter møtes (Eriksen, 2015: 889). Sykepleiers
holdninger, forventninger og åpenhet for informasjon har betydning for å bygge
og utvikle en menneske til menneske relasjon (Brataas, 2011: 89). Man må som
sykepleier forholde seg til pasientens erfaringer, med deres meninger, uttrykk
og følelser (Eriksen, 2015: 889). Relasjonen til sykepleier er viktig for
pasientens opplevelse av pleien. For at sykepleier skal kunne ivareta
pasientens psykososiale behov, må man ta utgangspunkt i den enkelte pasient.
Man må bruke god tid til å bli kjent med pasienten for å kunne avdekke hans
behov (Eriksen, 2015: 901).

I Travelbee sin definisjon av sykepleie sier hun at “Sykepleie er en
mellommenneskelig prosess der den profesjonelle sykepleiepraktikeren hjelper et
individ, en familie eller et samfunn med å forebygge eller mestre erfaringer
med sykdom og lidelse, og om nødvendig å finne mening i disse erfaringene”
(Travelbee, 2001: 29). Videre forklarer hun at det er en “mellommenneskelig
prosess” fordi det alltid dreier seg om mennesker, enten direkte eller
indirekte (Travelbee, 2001: 29-30).  Når humor blir brukt for å bygge
relasjoner, kan den være kontaktskapende (Tyrdal, 2002: 192). Arnold \&{} Boggs
skriver at det kan bidra til økt nærhet og det kan hjelpe med å styrke båndet
mellom pasient og sykepleier, dersom det allerede er etablert en god og trygg
kontakt (referert i Eide \&{} Eide, 2008: 247).  For en pasient kan det å motta
dårlige nyheter føre til psykisk smerte. I følge Kari Martinsen har omsorg med
relasjoner og moral å gjøre, dette viser seg igjennom praktisk handling. For at
sykepleier skal kunne hjelpe pasienten, må man  først og fremst kunne sette seg
inn i deres situasjon (referert i Reitan, 2008: 80-81).  Humor kan ha stor
positiv effekt, men det kan også brukes på en negativ måte for eksempel ved å
latterliggjøre eller kritisere. Dette bør ikke forekomme i relasjoner til
pasient. Spøk og fleip kan ha et skjult, negativt budskap (Eide \&{} Eide, 2008:
246). Om en situasjon oppleves som god eller dårlig av pasienten avhenger av
sykepleierens væremåte og holdning. Sykepleiers utfordring blir å møte
pasienten på en god måte, hvor moral og etikk viser seg gjennom væremåte,
kroppsspråk og det som blir sagt og gjort (Brinchmann, 2008: 127).

Å hjelpe et menneske til å mestre sykdom kan være vanskelig, å desto
vanskeligere dersom fremtiden er usikker. Her er det viktig med gode relasjoner
og kunnskap om hva den enkelte trenger for å finne mening når livet er
vanskelig. Travelbee (2001) sier at det er først når et godt samarbeid mellom
pasient og sykepleier er oppnådd at man kan hjelpe den syke med å finne håp og
mening, og hjelpe dem til å mestre sykdom og lidelse. Videre sier hun at en
sykepleier må strebe etter forandring, slik at pasientens helse opprettholdes
best mulig (Travelbee, 2001: 30).

\section{Humor og mestring}

Lazarus \&{} Folkman har utarbeidet en definisjon som sier at mestring er “stadig
skiftende kognitive og handlingsrettede forsøk som tar sikte på å håndtere
spesifikke ytre og /eller indre utfordringer som blir oppfattet som byrdefulle
eller som går utover de ressursene som personen rår over” (Heggen, 2010: 65).
Dette er en prosessorientert definisjon hvor det blir lagt vekt på hva
mennesket tenker og gjør, ikke hvordan mennesket er. Den er knyttet opp til en
konkret eller spesifikk utfordring, og skiller mellom handlingen og utfallet.
Mestring er i denne sammenheng en målrettet handling, som kan være både
vellykket og mindre vellykket (Heggen, 2010: 65). I følge Gjærum kan ikke
mestring sammenliknes med å fikse situasjonen. Det blir begrunnet med at
enkelte situasjoner ikke har noen eksakte løsninger, men mestringsmuligheter
(referert i Reitan, 2008:58).  Mestring handler om hvordan mennesker håndterer
belastende livssituasjoner. Om man ser enkelt på det, kan man si at det er to
måter å reagere på. Man kan enten mobilisere indre og ytre ressurser for å møte
utfordringen, eller man kan fortrenge eller skyve bort ubehaget ved hjelp av
ulike forsvarsmekanismer (Reitan, 2006: 157). Man må ta i bruk de ressurser man
har for å bedre situasjonen. Enten det er for å takle stress, påkjenninger,
kriser eller sykdom, slik at man kan komme seg videre på best mulig måte
(Heggen, 2010: 64).  Det er flere faktorer som påvirker menneskets evne til
mestring. Noen eksempler på dette kan være kjønn, alder og personlige
interesser. Tidligere erfaringer med sykdom, stress eller krise har også
betydning for evnen til å mestre (Reitan, 2008: 58).  Mennesker er humoristiske
individ (Eide og Eide 2008: 242). Borge \&{} Kristoffersen viser til en studie av
eldre, nyopererte pasienter som viser at pasientene syntes at sykepleierne
brukte for lite humor. I de tilfellene hvor humor ble brukt, var det pasientene
selv som tok initiativet.  Humor kan være viktig for pasientenes trivsel og det
kan brukes som en mestringsstrategi, både for pasient og sykepleier (referert i
Eide \&{} Eide, 2008: 242-243).  Mechanic skriver at humor kan være en god
mestringsstrategi (referert i Eide \&{} Eide; 2008: 244). Man kan aktivt bruke
humor for å mestre situasjonen man befinner seg i. Det kan brukes for å
beskytte seg selv i en tøff hverdag, men samtidig bør det ikke brukes for å
fornekte situasjonen. Man kan ofte observere at pasienter som spøker med
alvoret, reduserer spenningen i situasjonen (Wist, 2002: 164).  Ved alvorlig
sykdom kan man bli fokusert på egne problemer og lidelser, dette medfører at
evnen til å forholde seg åpent til andre kan være redusert. Humor kan i denne
sammenheng åpne opp, lette på stemningen og skape kontakt. Det kan også gjøre
det lettere å snakke om det som er vanskelig. Pasienten kan befinne seg i en
tragisk, fortvilende eller absurd situasjon, hvor god humor kan føre til
forsoning med dette. Humor kan gjøre det mulig å være nær det som er vanskelig,
om så bare for en liten periode (Eide \&{} Eide, 2008: 244-245). Lang mfl.
uttrykker at humor kan virke befriende fra den vanskelige situasjonen, ved at
det skaper en avstand fra alvoret (referert i Eide \&{} Eide, 2008: 245).

\chapter{Metode}

Vi har valgt å gjøre en litteraturstudie med kvalitativ metode som
utgangspunkt. En litteraturstudie er en studie hvor innsamlingsdata hentes fra
litteraturen, og man får en strukturert oversikt over et valgt tema (Segesten,
2006). Kvalitativ metode er en forskningsmetode som gir beskrivende data. Den
søker å gi en dypere forståelse og en økt kunnskap om et tema. Ved hjelp av
intervju og/eller observasjon får man innsikt i menneskers personlige
opplevelser og erfaringer (Olsson \&{} Sörensen, 2003: Sidetall). Vi har valgt å
analysere fem kvalitative forskningsartikler etter Fribergs modell i “Dags för
uppsats” (2006). Kvalitative studier gir en økt forståelse for hvordan man kan
møte pasientenes behov gjennom å se på pasientens opplevelser, erfaringer og
forventninger (Segesten, 2006: sidetal). Dermed er kvalitativ forskning den
metoden som er best egnet for besvarelse av vår problemstilling.  Vi ønsker å
se nærmere på hvordan man kan bruke humor i arbeid med kreftpasienter, hvordan
det blir mottatt, og vi ønsker å vite noe om hva pasientene opplever som
positivt og negativ ved bruk av humor. Gjennom grundig analyse av artiklene
finner vi gode beskrivelser av pasientenes og helsepersonells egne erfaringer,
opplevelser og atferd, gjort gjennom observasjoner og intervju.

\section{Litteratursøk}

Litteratursøket til oppgaven ble gjennomført i perioden 26.11.2015 til
03.12.2015. Databasene vi brukte til å søke opp artikler var Cinahl og Oria. Vi
har også brukt tidsskriftet Scandinavian Journal of Caring Sciences’ egen
søkemotor for vitenskapelige artikler. De fleste artiklene fant vi i databasen
Cinahl. Dette er en database som blir mye brukt til søk av vitenskapelige
artikler innen helsefag, og var derfor godt egnet for oss.  I Cinahl valgte vi
å begrense søket til å gjelde fra 2005 til 2015 slik at vi skulle få de senest
publiserte artiklene, og den nyeste kunnskapen. Vi huket av på ”full text” og
“peer reviewed” da dette indikerer at artiklene er vitenskapelige og
kvalitetssikret av andre. Søkeordene som ble brukt var engelske. Dette fordi de
fleste vitenskapelige publikasjoner utgis på dette språket. Grunnen til det er
at forfatteren vil nå ut til flest mulig med sin forskning (Friberg, 2006:
sidetall). Søkeord som Humor, Humour, Health care og Nurs ble brukt. Vi fant
fire artikler her som vi ønsket å undersøke videre.

I Oria huket vi av for “artikler” og brukte søkeordene “Humor” og
“Relationship”. Her fikk vi 1464 treff. Videre avgrenset vi for årstall i
tidsrommet 2005 - 2015, og at artiklene skulle være fagfellevurdert. Dette
førte til 584 treff. Blant dem fant vi én som vi ønsket å se nærmere på.
Artikkelen var tilgjengelig i fulltekst hos ProQuest Health and Medical
Complete.

Den siste artikkelen vi ønsket å se nærmere på fant vi i Scandinavian Journal
of Caring Sciences. Her ble søkeordene “humour” og “cancer” brukt. De fleste
artiklene ble forkastet da de ikke oppfylte våre kriterier, og av 17 treff var
bare én av disse relevant for oppgaven vår.

\subsection{Oversikt av analyserte artikler}

\todo{formatering}
Artikkel
	Cancer survivors’ experiences of humour.
(2015)
	A time to weep and a time to laugh.
(2013)
	More than trivial.
(2005)
	Humour in health-care
interactions.
(2011)
	From critical care to
comfort care.
(2007)
	Forfatter
	RoaldsenB,L., Sørlie T., Lorem G,L.
	Tanay M,A.,
Wiseman T.,
Roberts J.,
Ream E.
	Dean R,A.,
Gregory D,M.
	McCreaddie M., Payne S.
	Dean K., R,A.,
Major E., J.
	Tilnærming
	Kvalitativ studie
	Kvalitativ studie.
	Kvalitativ studie.
	Kvalitativ studie.
	Kvalitativ studie.
	Metode
	Intervju i perioden 2010-2011 av første forfatter.
	Intervju,
semi-
strukturerte intervju,
uformelle intervju og
observasjon
	Observasjon av sykepleiere over seks uker. Intervju med pasienter og pårørende.
Semi- strukturere intervju med helse- personell.
	Intervju,
observasjon
notater og
lyd-dagbok.
	Sammen-
ligning av to tidligere studier.
	Deltakere
	14 pasienter i alderen 23-83 år.
7 kvinner og 7 menn.
	9 syke-
pleiere og 12 pasienter ble observert.
5 sykepleiere og 5 pasienter ble intervjuet.
	6 sykepleiere
ble observert, 11 sykepleiere,
2 sosial-
arbeidere og 1 fysio-
terapeut
ble intervjuet.
	32 deltakere
4 pasient
fokus-
grupper.

	Søkemetode
	Søk i tidsskrift.
	Søk i database.
	Søk i database.
	Søk i database.
	Søk i database.
	

\section{Analyse}

Analyse av tekstene har vi gjort med utgangspunkt i Fribergs analysemodell i
“Dags för uppsats” (2006). Friberg mener det er viktig å fokusere på studienes
resultat, og lese gjennom artiklene flere ganger slik at man får ordentlig tak
på hva de handler om. På denne måten blir det enklere å plukke ut hovedfunn og
gjøre en sammenligning mellom de ulike artiklene. Friberg benytter en
hermeneutisk tilnærming i analysearbeidet, og beskriver det som en bevegelse.
Når man har valgt ut sine artikler, deler man artiklenes resultater inn i de
kategorier man søker. Deretter setter man det sammen igjen til et nytt
resultat. Her ser man at arbeidet går fra en helhet, til deler og tilbake til
en ny helhet (Friberg, 2006: sidetall).

Etter å ha lest gjennom abstraktet, valgte vi ut de artiklene vi syntes var
mest relevante. Det var vanskelig å finne artikler som omhandlet både kreft og
humor, og som samtidig holdt opp til våre kriterier. Derfor valgte vi å bruke
noen artikler som omhandlet palliasjon, da dette kan være en naturlig del av et
kreftforløp. Med hovedfokus på tema og resultat, leste vi gjennom alle
artiklene for å få en oversikt over hva de handlet om, og om de oppfulgte våre
kriterier. Noen av artiklene ble byttet ut underveis, da vi ikke syntes de ga
svar på det vi søkte etter. Deretter finleste vi artiklene for å finne likheter
og ulikheter mellom de forskjellige. Gjennom fargekoder og skjematisk
fremstilling sorterte vi de ulike funnene. På den måten fikk vi en god oversikt
over likheter og ulikheter i artiklene. Ut i fra dette kunne vi dele det videre
opp i hoved- og underkategorier. Dette var et krevende arbeid. Det var mange
interessante resultater, og mye vi kunne tenkt oss ha med, så det var
utfordrende å holde fokus på det vi søkte etter. På bakgrunn av funnene vi
gjorde, satte vi til slutt sammen to hovedtema med tilhørende  subtema.

\todo{hva skal dette være}
Bruk av humor i kommunikasjon

	* Relasjonsbygging mellom sykepleier og pasient
* Humorens motsatte effekt
* Den vanskelige samtalen
* Individuelle hensyn
	Bruk av humor som mestringsstrategi

	*  Håndtering av følelser
*  Tilbakevennende humor


\chapter{Resultat}

\section{Bruk av humor i kommunikasjon}

\subsection{Relasjonsbygging mellom sykepleier og pasient}

Artikkelen til Tanay el al. (2013: 1297) viste at både sykepleier og pasient så
på humor som et tegn på at det ble utviklet et personlig forhold mellom dem.
Flere pasienter opplevde at de hadde et godt forhold til sykepleier dersom de
lo sammen. Sykepleierne oppfattet viktigheten humor hadde for å etablere en god
relasjon til pasientene, og effekten det hadde for å få pasientene til å føle
seg mer avslappet. Ved bruk av humor fikk sykepleierne vise at de også var
mennesker, og dette gav en følelse av tilhørighet for pasienten og de rundt
dem. Det virket som om pasientene fikk større tillit til sykepleierne dersom
det ble brukt humor. På den måten ble det lettere for pasientene å åpne seg, og
snakke om de vanskelige tingene (Tanay, Wiseman, Roberts \&{} Ream, 2013: 1297).

I studien til Tanay et al. (2013: 1298) så det ut til at sykepleiere foretrakk
morsomme pasienter. Det var lettere for pasienten å ta kontakt med en
sykepleier som viste en form for humoristisk sans, i forhold til en som ikke
hadde det. En pasient sier at man ved bruk av humor hadde større sjanse for å
bli likt. Dersom man brukte humor til å spørre om noe, svarte sykepleierne
muntrere. En pasient beskriver det slik: “Somebody with a sense of humour
asking for a cup of tea, is more likely to get one than somebody demanding a
cup of tea”. (Tanay et al.,2013: 1298)

McCreaddie og Payne (2001) finner i sin studie at humor ikke alltid var
positivt. Noen pasienter kjente på behovet for å være en “god” pasient, og
prøvde dermed å adoptere sykepleierens væremåte. Dersom sykepleieren brukte
humor, gjorde også pasienten det for å oppnå bedre kontakt og for å få den
hjelpen de trengte. De fant også at pasientene brukte humor til å uttrykke
følelser og engstelse. Dette kunne være risikabelt da sykepleier ikke oppfattet
alvoret fra pasienten.

\subsection{Den vanskelige samtalen}

Sykepleiere mente at humor fikk pasientene til å føle seg hjemme og være mer
avslappet. De tok dette som en indikasjon på at pasientene stolte mer på dem
dersom de hadde ledd litt sammen. Dette førte til at pasientene åpnet seg mer
og ønsket å snakke om de seriøse tingene (Tanay et al., 2013:1297).  Humor ble
av pasientene brukt som beskyttelse når de for eksempel gikk tom for ord, eller
for å slippe å vise sårbarhet (Roaldsen et al., 2015: 4). Å sette ord på
erfaringer kunne være vanskelig. Det å bruke humoristiske uttrykk kunne
indirekte være en måte å kommunisere forståelig med andre. De brukte
humoristiske metaforer og bilder med varierende intensjoner, som for eksempel å
stanse sensitive temaer uten at samtalepartneren skulle føle seg avvist.
Sykepleierne sa at de ved å bruke humor på den måten kunne distansere seg selv
fra pasientene, og slippe å måtte gå inn i seriøse diskusjoner med dem (Dean \&
Major, 2007).

I Roaldsen et al. (2015) reagerte en pasient med latter da hun fikk diagnosen
brystkreft. Hun sa at dette var den eneste måten hun greide å reagere på, og
sa: “ You mustn’t laugh, because then they’ll think you’re crazy, ‘cause people
are supposed to cry” (Roaldsen et al., 2015: 4).  Innen palliativ pleie var det
ofte humor involvert når de snakket om fortiden. Mimring om fortiden var
spesielt viktig for pasientene da de ble konfrontert med at det gikk mot
slutten. Av og til delte de byrder fra fortiden, men ofte også høydepunkter
gjennom livet. Dette var også en mulighet for personalet å få god kontakt med
pasienten og deres pårørende (Dean \&{} Gregory, 2005).

\subsection{Individuelle hensyn ved bruk av humor}

I studien til Dean \&{} Gregory (2005) kom det frem flere faktorer som var
avgjørende for bruk av humor i sykepleien. I tillegg til etnisitet, kjønn og
stressnivå  var bruk av humor avhengig av pasientens personlighet og
situasjonen pasienten var i. I tillegg uttrykte noen sykepleiere en utrygghet
ved bruk av humor i jobben. De fryktet at det skulle gå utover deres
profesjonalitet og de var bekymret for hvordan deres kolleger skulle se på dem.
Dette kan være grunnen til at mange unge sykepleiere var mer seriøse på jobb i
møte med eldre sykepleiere. Når det bare var unge sykepleiere ble det observert
mer humor og tøys (Tanay et al., 2013:1298).

Det ble pekt på flere omstendigheter hvor deltakerne mente at humor ikke var
passende. Ved endring i pasientens tilstand kunne det være mye frykt, sinne og
sorg. I slike situasjoner ble forsøk på bruk av humor ikke satt pris på av
pasientene. Det kom frem flere situasjoner hvor pleierne forsto at de hadde
gått for langt, og at humor ikke var passende (Dean \&{} Gregory, 2005). I
McCreaddie \&{} Payne (2011) var det derimot enighet blant pasientene om at de
satte pris på sykepleiernes bruk av humor også i situasjoner hvor man ikke
skulle tro det var passende. Pasientene syntes det var positivt dersom
sykepleierne var muntre og glade. Det kunne bidra til å lette på stemningen i
vanskelige situasjoner. Pasientene syntes også at det var bra at sykepleierne
tok noen sjanser, og at de ikke var for redde for å bruke humor.

I situasjoner hvor pasienten lå på dødsleiet var det av den oppfatning at
humoren skulle bli overlatt til pasienten og pårørende. Det var her ikke
passende for personalet og komme med kommentarer av humoristisk karakter (Dean
\&{} Gregory, 2005). En sykepleier fortalte om rørende øyeblikk rett før døden.  “
in their last minutes of life I’v seen humour used there too. It’s a very
loving humour, it’s kind of heart-to-heart humour from a family member to the
one who’s dying” (Dean \&{} Gregory, 2005: 296).

Både pasient og sykepleier mente at det var viktig å ta hensyn til etnisitet.
Det var likevel vanskelig for deltakerne i undersøkelsen til Dean \&{} Gregory
(2005) å gi noen spesifikke eksempler på hensyn man måtte ta, utenom at det var
et behov for sensitivitet og forsiktighet rundt bruk av humor. En pasient mente
at humor var vanskelig mellom mennesker fra forskjellige kulturer. Språket
gjorde at det kunne oppstå misforståelser som følge av feil uttalelser eller at
ting ble feiltolket.  Det var forskjell i hva kvinner og menn foretrakk ved
bruk av humor. Menn hadde en tendens til å bruke humor som et middel for å være
åpne med hverandre og for å dekke over ubehag. Deres bruk av humor var også mer
preget av seksuelle bemerkninger over for personal, noe som gjorde at mange
kvinnelige sykepleiere følte seg utilpass. Noen håndterte dette ved å overse
kommentarene, mens andre vitset det bort (Dean \&{} Gregory, 2005).

I Dean \&{} Gregory (2005) hadde deltakerne vanskelig for å svare på når og
hvordan de brukte humor. Mange svarte at det ikke var noe bevisst, men at det
bare oppsto spontant. Andre igjen sa at det var vel overveid og at timingen
måtte være rett. I Studien til Tanay et al. (2013) sa noen at de hadde en
intuisjon om når det passet. Humor er individuelt og det er forskjeller i
hvordan det blir mottatt. For å ikke ødelegge sykepleier - pasientrelasjonen
var det viktig at man vurderte situasjonen og så etter tegn fra den andre
personen før man brukte humor.

\section{Mestring}

\subsection{Bruk av humor som mestringsstrategi}

I artikkelen til Roaldsen et al. (2015) snakket noen pasienter om humor som kom
og gikk. I perioder av sykdommen som var preget av mye uvisshet, angst og
stress kunne sansen for humor forsvinne. I bedre perioder kunne humoren komme
tilbake. Man kunne da ofte se at typen og bruken av humor var noe endret. De
brukte for eksempel mer galgenhumor og ironi rettet mot sin egen situasjon, og
sine erfaringer med kreftsykdom. Denne typen humor ble gjerne brukt som en
slags beskyttelse for å dekke over hvor vanskelig de egentlig hadde det. En
pasient beskriver det slik: “That’s how you blossom. Humour keeps away those
heavy thouhts. Yes, it gets more like gallows humour. You use it, quite
obviously, to put thing at a distance” (Roaldsen et al., 2015: 5).

En annen pasient sa at galgenhumor var en måte å beskytte seg selv ved å dekke
over det hun virkelig følte, som var hvor fæl denne erfaringen hadde vært. “...
if I couldn`t have laughed and had fun with it, well I think I’d have had to go
through a very dark time, I mean, I don`t think I`d have managed to be so
strong” (Roaldsen et al., 2015: 5).

I samme artikkel sa andre at i perioder med kaos og fortvilelse, var humor
særlig viktig for å håndtere følelsene. På den måten overtok ikke sykdommen all
plass. Humor hjalp pasientene til å distansere seg fra negative følelser og
tunge tanker. Dette kunne de gjøre ved å se morsomme tv program, eller
underholdende klipp på YouTube. Dette lot dem få avstand fra ensomheten og
tankene om døden. Humor og latter var viktig både for helsepersonell og
pasientene for å håndtere spenningen og tristheten som ofte dukket opp. Dette
skapte pusterom fra de tunge situasjonene de befant seg i.

Under kreftbehandling fortalte pasientene at deres mestring av situasjonen ble
satt på prøve. De var derfor bevisste sine mestringsstrategier og understreket
at humor var viktig for å skape en levelig situasjon. Muligheten til å
akseptere og tåle sykdom var knyttet til et humoristisk livssyn som gav
opplevelse av mening, og det å være deltakende i samfunnet. Det betydde at man
måtte takle motgang og akseptere sykdommen som et faktum. Pasienter sa at det
var viktig i denne sammenheng å finne en meningsfull balanse der sykdommen ikke
ble for dominerende (Roaldsen et al., 2015).

Flere deltakere foralte at deres bruk av humor var påvirket av graden av stress
de erfarte. Det interessante her var at enkelte brukte mer humor jo høyere
stress de erfarte. I kontrast, var andre mindre mottakelige for bruk av humor
ved mye stress (Dean \&{} Gregory, 2005).

Humor kunne gi et lysere perspektiv på en vanskelig situasjon. Ved å overføre
en tragisk situasjon til underholdende ord, kunne man le av fortvilelse, sinne
og sorg. En pasient i Roaldsen et al. (2015) bestemte seg for å leve i
øyeblikket, da han fikk vite at det var fare for spredning av sykdommen.

\todo{formatere sitat}
 ...then I sat for a long time thinking: What if I die? The kids. What about
 them? My Wife, the money, the house? But suddently I thought: No, bloody hell!
 there is another alternative, and that`s that everyting`s fine! You can`t bury
 yourself in seriousness, then you might as well close the lid… (side 6)

Videre sa han at humor kanskje ikke hjelper deg til å overleve, men at det kan
gjøre livet bedre oppi alt annet.

I palliativ pleie, hvor døden kunne være nær, hadde følelser en tendens til å
være forsterket. Humor kunne da maskere underliggende følelser. Dette kunne
sende ut feil signaler og føre til at pasientene ikke fikk den hjelpen de
trengte. De erfarne sykepleiere som jobbet rundt disse pasientene, lærte seg
etterhvert å se hva som egentlig  gjemte seg bak humoren (McCreaddie \&{} Payne,
2011).

\chapter{Diskusjon}

\subsection{Metodediskusjon}

Vi ønsket å se nærmere på bruk av humor som kommunikasjonsmetode og som
mestringsstrategi i sykepleien. Vi ønsket å få et innblikk i hvordan pasientene
opplever bruk av humor, hvordan sykepleiere bruker humor i sitt arbeid og
hvordan pasientene bruker humor til å mestre hverdagen. Å gjøre en empirisk
undersøkelse ville være altfor tidkrevende for oss å utføre. Vi valgte derfor å
gjøre en litteraturstudie hvor det er gjort bruk av kvalitative
forskningsartikler. Kvalitativ metode gir oss et perspektiv sett innenifra, og
gir en dypere forståelse for et fenomen blant annet gjennom intervjuer og
observasjon. Man får på den måten et bedre innblikk i menneskers tanker,
opplevelser og følelser (Olsson \&{} Sörensen, 2003: sidetall). Det kan være
vanskelig å få et bilde av tanker og følelser gjennom tall og statistikk. Vi
syntes derfor at kvalitativ metode var bedre egnet enn kvantitativ metode i
denne oppgaven.

Litteraturstudier innhenter data fra allerede analysert materiale som
vitenskapelige artikler (Friberg, 2006: sidetall). Metoden har blitt kritisert
fordi den ikke kommer frem med ny forskning, at det er for lite utvalg av
materiale og at den kan bli for subjektiv. Forskerne eller forfatterne leter
etter det de ønsker å finne, dermed kan kanskje andre relevante funn kan bli
oversett (Friberg, 2006).

Det er blir gjort relativt lite studier om temaet humor i sykepleie og utvalget
av artikler som var relevante for vår problemstilling var dermed begrenset.
Bare én av de fem artiklene vi valgte ut er norsk, mens de resterende er fra
engelskspråklige land. Vi fant svært få undersøkelser som var gjort i Norge
eller Skandinavia forøvrig, men vi tenker at samfunn og kultur er såpass likt,
at artiklene likevel kan overføres til norske forhold. Vi har ikke sett på
studier hentet fra andre deler av verden, og det kan vel derfor tenkes at denne
studien blir rettet mot den vestlige verden. Man kan jo anta at man i andre
deler av verden har kulturforskjeller hvor man har et litt annet syn på bruk av
humor enn oss, noe som igjen ville gitt oss et litt annet resultat. Selv om det
var et begrenset utvalg artikler, besvarer disse likevel vår problemstilling og
vi ser at resultatene i artiklene er gjenkjennelige fra egen praksis og
erfaringer.

Artiklene er hentet fra tre ulike databaser og er fra forskjellige land.
Likevel ser vi at det er mye de samme temaene som går igjen, og at både hensikt
og resultater samsvarer ganske mye. Dersom vi hadde utvidet søket og hatt enda
flere artikler kan det hende at vi hadde fått et litt bredere resultat enn det
vi har. Da resultatene likevel virket å være såpass like, valgte vi heller å
plukke ut noen få artikler, og fokusere på noen tema som gikk igjen.

Før vi startet søket på artikler, undersøkte vi litt rundt i litteraturen
generelt for å se hva vi fant rundt temaene “humor” og “kreft”. På bakgrunn av
dette dannet vi oss et bilde av hva vi ønsket å se nærmere på, og det gav oss
en pekepinn på hvilke søkeord vi skulle bruke. Underveis i analysearbeidet
endret fokus seg noe, og tema som “humor og fysiologisk betydning” ble for
eksempel valgt bort, da vi heller ønsket å rett fokus mot kommunikasjon og
mestring.  Tre av artiklene er av nyere dato, mens de to andre er noen år
eldre. Vi valgte likevel å ta disse med da vi ser at de er like dagsaktuelle
som de andre.  I fire av artiklene er datainnsamling hentet inn via intervju og
observasjon, mens den siste artikkelen er basert på funn fra sammenligning av
to vitenskapelige artikler. Det er gjort både semistrukturerte og uformelle
intervju. I den ene artikkelen er det kun hentet inn data via intervju, ikke
observasjon. I samtlige artikler hvor data er hentet inn via intervju og
observasjon er informantene helsearbeidere og pasienter. I den ene studien er
det også pårørende som informanter, men disse har vi midlertidig ikke valgt å
legge vekt på da vårt fokus ligger på pasient og sykepleier.

I kvalitativ metode skal man kunne gå i dybden og få en nærhet til det man
ønsker å undersøke. Det er da en forutsetning at man har et mindre antall
informanter (Olsson \&{} Sörensen, 2003: sidetall). I artiklene vi har valgt
gjenspeiles dette ved at  det hovedsakelig er små grupper på mellom 20 - 30
informanter. Ved et lite antall informanter er det enklere for forskeren å få
en relasjon til vedkommende og på den måten få et bedre innblikk fra
informantens perspektiv.

\section{Resultatdiskusjon}

Vi vil besvare problemstillingen vår ved å sette resultatene opp mot teori og
egne synspunkter. Vi har delt drøftingen i to. I første del ser vi på hvordan
humor i kommunikasjon brukes i relasjonsbygging og i den vanskelige samtale,
samt hvilke hensyn som bør tas ved bruk av humor. I del to om mestring ser vi
på hvordan pasienten kan bruke humor som en mestringsstrategi for å håndtere
hverdagen som kreftsyk.

\subsection{Bruk av humor i Kommunikasjon}
\todo{skal det være para her?}
Relasjonsbygging mellom sykepleier og pasient

Humor kan i følge Arnold \&{} Boggs være en nyttig og virkningsfull
kommunikasjonsstrategi. Det kan være viktig for pasientens trivsel og kan
brukes til å redusere stress og spenning (referert i Eide \&{} Eide, 2008) I
sykepleiepraksis er relasjonen mellom sykepleier og pasient viktig. Vi vet at
det er mye lettere å få til et godt samarbeid med pasienten dersom man har en
god relasjon. Som sykepleier må man forholde seg til pasientens erfaringer og
de meninger, utrykk og følelser som følger med dette. Vi vet fra tidligere
erfaring at relasjon mellom sykepleier og pasient er viktig for hvordan
pasienten opplever pleien, dette blir også understrekt av Eriksen (2015).

Tanay et al. (2013) viser eksempler på at både sykepleier og pasient tar hensyn
til hverandre. En sykepleier fra studien sa: “I follow their (patients) lead…”
(Tanya et al., 2013: 1299). Det kommer frem av denne studien at pasientene så
viktigheten av refleksjon og vurdering rundt sin humorbruk. En pasient sa: “so
I`d perhaps wait a bit and find out how she (nurse) was, let her make the next
move, basically” (Tanay et al., 2013:1299).  Dette kan kanskje tolkes som at
begge er forsiktige med bruk av humor for de ikke har blitt helt komfortable
med hverandre og har vanskelig for å lese hva den andre synes er akseptabelt.

I artikkelen til Tanay et al. (2013) ser vi at både pasient og sykepleier så på
bruk av humor som et tegn på at det ble utviklet et personlig bånd mellom dem,
noe som førte til at pasientene følte seg mer komfortable sammen med
sykepleier. Man ser altså at man bør ha et visst forhold til en person, for å
føle seg komfortabel nok til å bruke humor. Tyrdal (2002) og Arnold \&{} Boggs i
Eide \&{} Eide (2008) understreker dette. Humor kan skape bedre kontakt mellom
sykepleier og pasient, dersom trygg relasjon allerede er til stede.  Travelbee
(2001) sier at det er først når pasient og sykepleier har et godt samarbeid at
sykepleier kan hjelpe og ivareta pasientens behov. I artikkelen til Tanay et
al. (2013) kom det frem at sykepleierne forsto at humor kunne hjelpe dem å
skape denne relasjonen, og at det førte til at pasientene følte seg mer
avslappet. Ved å bruke humor fikk sykepleierne vise at de også bare var
mennesker.  I Tanay et al. (2013) så det ut til at sykepleiere foretrakk
morsomme pasienter, noe som kan ha negative konsekvenser for pasientene. En
pasient i denne studien sa at dersom man brukte humor, var sjansen større for å
bli likt av sykepleierne og at man fikk den hjelpen man trengte. Dette ser også
vi på som negativt. Vi mener at man som sykepleier skal være profesjonell og
ikke la personlige tanker eller fordommer påvirke den hjelpen som gis. Alle
skal få den hjelpen de trenger, uansett hvem de er. I McCreaddie \&{} Payne (2011)
kom det også frem at pasientene følte de måtte innynde seg hos sykepleierne for
å få den hjelpen de trengte.

\todo{para?}

Humorens motsatte effekt

Humor ble stort sett beskrevet som positivt av deltakerne i studien til
McCreaddie \&{} Payne (2011) fordi det kunne være en god hjelp i å lyse opp
hverdagen og håndtere stress. Som vi tidligere har sett er ikke humor en
løsning i seg selv, men den kan fungere som en beskyttelse og et kortvarig
pusterom (Reitan 2008, Wist 2002). Bruk av humor kan være tvetydig. Den kan
brukes både til å maskere underliggende følelser og for å uttrykke bekymringer.
Dette kommer frem i studien til McCreaddie \&{} Payne (2011) som problematisk
dersom pasientene sendte ut signaler om hjelp i form av humor, men som ikke ble
oppfattet slik av sykepleierne. Det kunne også være et problem den andre veien,
dersom en spøk ble tatt alvorlig av sykepleier. Grunnen til at slike
situasjoner oppstod kunne være at pasient og sykepleier hadde ulik sans for
humor. Hvordan humor ble tolket var viktig i forhold til om pasienten fikk den
hjelpen de søkte. Man ser at det i slike situasjoner er viktig at man som
sykepleier er oppmerksom og ser pasienten som en helhet. Man må prøve å lese
pasienten, ikke bare ut i fra hva som blir sagt, men også det non-verbale og
kroppslige. For å kunne gi god sykepleie er relasjonen til pasienten
avgjørende, og man ser fra eksempelet over at kjennskap til pasienten kan gjøre
det enklere å avsløre om det ligger et budskap gjemt i humoren. Travelbee
(2001) sier at gode ferdigheter innen kommunikasjon er noe som kan læres, og
kan hjelpe oss i arbeidet med å forstå og møte pasientens behov.

\todo{para?}
Den vanskelige samtalen

Vi vet fra egne erfaringer at samtaler med kreftpasienter kan være vanskelige.
Vi har også merket oss at pasienter åpner seg i større grad dersom vi alt har
etablert en god relasjon. Likevel har vi opplevd at pasienter bruker humor for
å unngå de tunge samtalene. Dette kommer også til syne i Roaldsen et al. (2015)
hvor pasientene brukte humor som en beskyttelse når de gikk tomme for ord,
eller ikke ville vise sårbarhet.

Å sette ord på erfaringer kan være vanskelig. I Dean \&{} Major (2007) kommer det
frem at pasientene brukte humoristiske utrykk som metaforer, for å kommunisere
forståelig med andre og for å unngå vanskelige samtaler. Dette ser vi et
eksempel på i Roaldsen et al. (2015) der en pasient sa at hun brukte
kallenavnet “naked rat” om seg selv da hun var hårløs. Dette brukte hun for å
beskytte seg selv. Wist (2002) beskriver at humor kan brukes for å dekke over
usikkerheter rundt egen situasjon. Vi har alle opplevd tilfeller hvor pasienter
bruker morsomme kallenavn om seg selv. I slike tilfeller er det ofte
usikkerheten på seg selv som ligger bak. I følge Mechanic kan en slik humor
være negativ fordi det kan oppfattes som kunstig munterhet for å unngå seriøse
samtaler (referert i Eide \&{} Eide, 2008). Studien til McCreaddie \&{} Payne (2011)
viser at det kan være risikabelt for en pasient å bruke humor for å uttrykke
følelser, da sykepleier ikke alltid oppfattet alvoret. Dette vil kunne påvirke
den hjelpen som pasienten mottar i negativ forstand. Det er viktig at vi som
sykepleiere klarer å se alvoret bak spøken. Roaldsen et al. (2015) viser et
eksempel hvor en pasient reagerte med latter da hun mottok sin diagnose. Dette
var hennes måte å håndtere situasjonen og uttrykke følelsene på. Hun var
likevel klar over at sykepleierne kunne oppfatte henne som gal siden hun ikke
gråt.  En alvorlig sykdom som kreft kan være altoppslukende, og man kan lett
bli fokusert og opptatt av egne problemer og lidelser. Dette kan medføre at
evnen til å forholde seg åpen til andre blir redusert. Vi har tidligere sett at
humor kan være kontaktskapende, og i den sammenhengen kan humor være til hjelp
ved at det åpner opp og letter på stemningen. Det kan bidra til bedre kontakt
med andre, og gjøre det lettere å snakke om det som er vanskelig (Eide \&{} Eide,
2008).  Det kan være ubehagelig å gå inn i vanskelige samtaler. Vi kan være
usikre på hva vi skal si eller gjøre, eller kanskje vi er redde for hvordan
pasienten opplever oss. Reitan (2008: 81) mener at det må være rom for dette.
Vi synes ikke at det bør gjøres som i studien til Dean \&{} Major (2007), hvor
sykepleiere brukte humor for å distansere seg fra pasienten, for å slippe å gå
inn i samtalen. Dette anser vi å være en negativ bruk av humoren. Vi tror at
dette kan føre til en dårligere relasjon mellom sykepleier og pasient. Det er
viktig at sykepleiere tørr å møte vanskelige situasjoner, men samtidig være
ærlige om ubehaget.  En positiv side ved bruk av humor er at det kan føre til
tillit mellom pasient og sykepleier. Dette kommer frem i studien til Tanay et
al. (2013) hvor pasientene hadde lettere for å åpne seg og snakke om de
vanskelige tingene dersom humor ble brukt i samtalene. Dette understrekes av
Arnold \&{} Boggs som i tillegg påpeker at det kan redusere vanskelige følelser og
lette på stemningen (referert i Eide \&{} Eide 2008). Videre i Tanay et al. (2013)
ser vi at sykepleierne mente at humor fikk pasientene til å føle seg mer
avslappet og hjemme.

\todo{para}
Individuelle hensyn

I følge Arnold \&{} Boggs er det ingen faste regler å forholde seg til ved bruk av
humor (referert i Eide \&{} Eide, 2008). Både Arnold \&{} Boggs og Bjørk Breievne
(2011) legger allikevel vekt på at det er forskjellige hensyn en må ta. Dette
synes også vi er viktig å ha fokus på, slik at humoren ikke blir fornærmende
eller får motsatt effekt enn planlagt. Arnold \&{} Boggs mener det er viktig at
man først vurderer personen og situasjonen man befinner seg i (referet i Eide \&
Eide 2008). I Bjørk \&{} Breievne (2011) kommer det frem at humor er personlig og
individuelt. I Dean \&{} Gregory (2005) pekes det også på at man skal ta hensyn
til ulike faktorer. Noe av det som nevnes her er etnisitet, kjønn, stressnivå,
pasientens personlighet og situasjon. En sykepleier sier i et intervju “humour
is like individuals, everyone`s different” (Dean \&{} Gregory, 2005:5). Et annet
eksempel i artikkelen viser at humor ofte blir dårlig mottatt av pasientene i
situasjoner hvor pasientens tilstand forverres. Dette tror vi kommer av at
pasienten i en slik situasjon kan oppleve mye angst i forhold til det som
skjer, og frykt for fremtiden. Wist (2002) peker i denne sammenheng på at man
kan la pasienten bruke humor omkring sin situasjon, men at man som sykepleier
er mer tilbakeholden med egne humoristiske bemerkninger. Som man kan se av
eksemplene passer det seg ikke alltid med humor i profesjonelle relasjoner.

Noen typer humor kan oppleves krenkende, noen eksempler her kan være sarkasme
og ironi. Det er viktig at man er sikker på at pasienten setter pris på denne
type humor (Eide \&{} Eide, 2008: 246-247). Humor mellom pasient og sykepleier
skal kunne deles, man skal kunne le sammen og ikke av (Wist, 2002: 166).

Vi tror at både pasient og sykepleier kan gå over grensen i forhold til hva som
regnes som akseptabel ved bruk av humor. I Dean \&{} Gregory (2015) nevner
pasientene at det er flere situasjoner der de forstår at de hadde gått for
langt, og at humor ikke var passende. Videre kan vi se at de hadde
vanskeligheter med å svare på spørsmål om når og hvordan de brukte humor. I
Tanay et al. (2013), svarte flere av deltakerne at det ikke var bevisst, men at
det var noe som oppstod spontant. Andre igjen svarte at humorbruken var godt
gjennomtenkt.  Humor er som tidligere sagt forskjellig fra person til person,
og det er derfor varierende hvordan den oppfattes (Bjørk \&{} Breievne, 2011). For
å bevare relasjonen mellom pasient og sykepleier, mener deltakerne i McCreaddie
\&{} Payne (2011) at man bør vurdere situasjonen og se etter tegn fra den andre
før man bruker humor. Videre viser denne artikkelen at pasientene syntes det
var positivt at sykepleierne var muntre og glade, og at de brukte humor. Dette
er noe vi også har erfart. Dersom man møter pasientene med et smil og en
positiv innstilling får man bedre respons. Det var enighet blant pasientene om
at de satt pris på humor, selv i de situasjonene enn skulle tro det ikke var
passende. Det ble omtalt positivt dersom sykepleierne tok sjanser og ikke var
redde for å bruke humor (McCreaddie \&{} Payne, 2011). Dette kan være litt
motstridene i forhold til Arnold og Boggs som mener at humor er uegnet før man
kjenner pasienten relativt godt (referert i Eide \&{} Eide, 2008). Dette tolker vi
som at man skal være sikker på pasienten og ikke ta noen sjanser, da det kan
ødelegge relasjonen. Wist (2002) mener også at vi som sykepleiere skal være
tilbakeholdne med egne humoristiske bemerkninger, men samtidig la pasienten
bruke humor omkring sin egen situasjon.

\subsection{Bruk av humor som mestringsstrategi}

\todo{para}
Håndtering av følelser

Mestring handler om hvordan et menneske håndterer en belastende situasjon
(Reitan, 2006). I vår oppgave vil det si hvordan den kreftsyke håndterer sin
sykdom og livet med sykdommen.  Perioder med kaos og fortvilelse kan kan være
utfordrende, og det kan være vanskelig ikke å miste seg selv. I Roaldsen et al.
(2015) forteller pasienter at det i slike perioder var særlig viktig å håndtere
følelsene og skape en balanse slik at sykdommen ikke tok opp all plass. De
kunne på den måten distansere seg fra de negative tankene og skape et pusterom
fra situasjonen de befant seg i. Under et kreftforløp er det viktig å prøve å
skape en levelig situasjon, og få en opplevelse av mening og håp. I artikkelen
til Roaldsen et al. (2015) brukte pasientene bevisst humor i forsøket på å
akseptere sykdommen slik at det var lettere å leve med den. Et eksempel som
beskriver dette var en pasient som fortalte at han satte mer pris på livet nå
enn før. Han kunne ofte mimre tilbake til hendelser som han da hadde opplevd
som forferdelige, men som han nå kunne le av, sett i sammenheng med situasjonen
han nå befant seg i. I følge Eide \&{} Eide (2008) kan humor hjelpe pasienten til
forsoning med den situasjonen man befinner seg i, om så bare for en kort
periode.

Flere pasienter forteller at deres bruk av humor var påvirket av graden stress
de erfarte. Man kunne kanskje tro at jo mer stress en pasient opplevde, desto
mindre humor ville de bruke. Studien til Roaldsen et al. (2015) viste et
eksempel av det motsatte hvor pasientene fortalte at de brukte mer humor dersom
stressnivået var veldig høyt. Fordi nivået av stress føltes så uoverkommelig
var dette deres måte å takle situasjonen på. Reitan (2006) beskriver denne
måten å håndtere stress på som en form for forsvarsmekanisme hvor man egentlig
bare skyver bort problemene. Selv om dette ikke var en problemløsende
håndtering fortalte pasientene at de ved å unngå at de tunge tankene og
følelsene fikk komme frem, var det enklere for dem å håndtere omstendighetene
rundt. Som Heggen (2008) påpeker er ikke humor en løsning på problemet, men man
tar i bruk de ressursene man har for å gjøre ting litt mer levelig. Vi har et
uttrykk som heter “en god latter forlenger livet”. En pasient i studien til
Roaldsen et al. (2015) beskriver det nettopp slik. Hun fortale at for henne
fungerte humor som medisin og en beskyttelse mot sin sorg og tristhet
forårsaket av sykdommen.

\todo{para}
Tilbakevennende humor

I artikkelen til Roaldsen et al., (2015) snakket pasientene om humor som
forsvant og humor som returnerte. Selv om noen pasienter brukte mer humor i
stressende situasjoner, fortalte andre at graden av humor kunne variere gjennom
sykdomsforløpet, og at den kunne forsvinne helt. I perioder hvor hverdagen var
preget av usikkerhet rundt fremtiden og omstendigheter rundt familien kunne
hverdagen bli en så stor psykisk påkjenning at mottakeligheten for humor kunne
forsvinne helt. Når situasjonen så bedret seg opplevde de at humoren kom
tilbake, men at den var endret. Humoren var da gjerne av den mørkere typen som
galgenhumor og ironi.  Dette er en form for mestringsstrategi som man kan se
hos pasienter som har det vanskelig. Man ser gjerne at pasientene spøker bort
alvoret som en måte å beskytte seg selv, og for å dekke over hvor vanskelig de
egentlig har det (Wist, 2002). En pasient forteller om hvordan hun følte det
når hun mistet håret som følge av kreftbehandlingen. I denne situasjonen
fortale hun at kroppen føltes både som en trussel og en støtte. For henne var
det å miste håret det verst tenkelige som kunne skje, og hun følte at bruk av
metaforer og selvironi var det eneste som kunne hjelpe henne å mestre
følelsene. Ved å beskrive seg som “the naked rat” og “Dumbo” satte hun opp en
beskyttelse for å ikke gå inn i en dyp sorg (Roaldsen et al. 2015).

\chapter{Anvendelse av resultatene i praksis}

Humorbruk kan være et godt hjelpemiddel i praksis. Gode grunner for anvendelse
av humor er at det kan føre til trivsel og glede, samt skape et godt samarbeid
mellom pasient og sykepleier. Man må ta hensyn til pasienten, og tolke deres
signaler for hva som er akseptabelt. Vi skal bruke humor som er til glede for
pasienten, ikke på deres bekostning. I tillegg vet vi at flere pasienter setter
pris på humor og det at sykepleier tar noen sjanser. Vi håndterer våre følelser
og hverdag på ulike måter, og bruk av humor kan for noen være en strategi for å
mestre dette. Som sykepleiere må vi forsøke å se om det ligger en dypere mening
under humoren.

\chapter{Avslutning}

Humor har alltid blitt omtalt som livets medisin, og en viktig helsefaktor. Det
har blitt gjort noen få vitenskapelige undersøkelser vedrørende humorens effekt
i arbeidet med syke, men det man ser er at bruk av humor berører flere sider
ved et menneske. Vi ønsket å se nærmere på hvordan humor ble brukt i relasjon
mellom sykepleier og pasient, og hvordan pasienter bruker humor som en form for
mestringsstrategi. Vi fant at humor ble brukt som et kommunikasjonsmiddel for å
bygge relasjoner mellom sykepleier og pasient, og at begge så på bruk av humor
som et tegn på at det ble utviklet et godt forhold mellom dem. Ved å ha en god
relasjon økte pasientens tillit til sykepleier, og man fikk større åpenhet og
bedre samtaler. Noen pasienter brukte humor som en form for mestringsstrategi,
hvor humor ble brukt for å beskytte seg, eller som en måte å takle hverdagen på
når ting ble ekstra tungt. Bruk av galgenhumor, ironi og metaforer ser vi som
eksempler på det. Humor ble stort sett oppfattet som positivt både blant
sykepleierne og pasientene. Pasientene syntes det var bra med litt munterhet i
alt det triste, og så på det som positivt hvis sykepleierne våget å ta noen
sjanser. Det ble også pekt på at humor må brukes med varsomhet, og at det ikke
passer seg like godt i alle situasjoner. Sykepleierne var i noen situasjoner
litt for tilbakeholdne med bruk av humor. Både fordi de var redde for hva
medarbeidere ville tenke, at situasjonen ikke var passende og at det ikke ville
bli godt mottatt av pasientene.
