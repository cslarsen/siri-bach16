\chapter{Teori}

Vi har delt teorien opp på følgende måte: I første del skriver vi litt generelt
om kreft, krise og livskvalitet. Videre ser vi på kommunikasjon og relasjon med
humor som virkemiddel. Her har vi fokus på forholdet mellom pasient og
sykepleier. Avslutningsvis ser vi på hvordan humor blir brukt som en
mestringsstrategi ut i fra et pasientperspektiv. På grunn av Travelbees' fokus
på det mellommenneskelige i utøvelsen av sykepleie, har vi har valgt å ha med
hennes perspektiv på de ulike temaene.

\section{Kreft}

Kreft har blitt en folkesykdom som rammer en stadig større del av den norske
befolkningen. I dag er det flere som lever lenge med sin kreftsykdom. Hos mange
utløser ordet kreft sterke assosiasjoner til døden. Det er en skremmende sykdom
som kommer innenfra ens egen kropp. Flere av de som rammes av kreft kan oppleve
at livet blir forandret eller forkortet. Flere kreftsyke ønsker ikke å bli
assosiert med sykdommen, da de føler at de i offentligheten <<blir>> sin kreft
\cite[s.~8--10]{amas2007}.

\subsection{Krise}

Et menneske med kreft står overfor store utfordringer, som kan innebære angst,
uro og bekymringer for egen livssituasjon. De kan oppleve en krise. Kjennetegn
på en krise, er at personens tidligere erfaringer og reaksjoner ikke er nok til
å forstå og mestre situasjonen de befinner seg i. Kriser har forskjellige
styrker og intensitet, og man opplever ulike grader av problemløsning.
Helsepersonell må kjenne til psykiske reaksjoner ved livstruende sykdom for å
kunne hjelpe pasientene. Å bli kreftsyk og det å gå gjennom behandling er ofte
forbundet med trusler om tap på flere områder. Det kan være tap av livet,
kroppens integritet, sosiale roller, aktivitet, selvoppfattelse, følelsesmessig
likevekt eller det å tilpasse seg en ny livssituasjon. Pasientene kan ha behov
for å uttrykke sine tanker, selv om det kan være vanskelige å verbalisere
\cite[s.~48--53]{reitan2008.kommunikasjon}.

\subsection{Livskvalitet}

Livskvalitet er et viktig begrep innen kreftomsorgen, da kreft er en sykdom som
rammer hele mennesket. Vi takler det å bli syk på ulike måter. Samme sykdom kan
gi ulike reaksjoner hos forskjellige mennesker. Dette kan avhenge av hvilke
fase i livet man befinner seg i, det sosiale nettverket rundt en og tidligere
erfaringer \cite[s.~39]{rustoen2008}.

Livskvalitet rommer det fysiske, psykiske og sosiale aspekt ved livet (Rustøen,
2008: 39). Den norske psykologen Siri Næss har utarbeidet en definisjon på
livskvalitet. Hun hevder at: \textquote[{sitert i
\citeNP[s.~39]{rustoen2008}}]{\ldots et menneske har det godt og har god
livskvalitet dersom man er aktiv, har samhørighet med andre, har god
selvfølelse og en grunnstemning av glede.}

Noe som er viktig for livskvaliteten er håpet, og da særlig håpet knyttet til
fremtiden. En av grunnene til at kreftsyke kan oppleve mye lidelse, er på grunn
av usikkerhet rundt fremtiden. Det å hjelpe pasienten med å bevare håpet er noe
sykepleier kan bidra med. Her spiller atferd og tilstedeværelse en viktig
rolle. En viktig faktor for å bearbeide kriser, er å finne mening i
situasjonen. Travelbee gjør rede for metoder som sykepleiere kan bruke for å
hjelpe pasienten til opplevelse av en meningsfull tilværelse. En av disse
metodene går ut på å hjelpe pasientene å innse at alle mennesker opplever
lidelse gjennom livet. Dette kan hjelpe pasientene til å komme over det første
sjokket, og til å starte med bearbeidelse av krisen
\cite[s.~40--42]{rustoen2008}.

\section{Humor i kommunikasjon}

Man bruker kommunikasjon for å komme i kontakt med andre mennesker, utveksle
tanker og følelser. Vi kommuniserer både verbalt og non-verbalt. Non-verbalt
gjør det lettere å forstå den verbale kommunikasjonen, og kan for eksempel være
ens kroppsholdning, kroppsspråk og ansiktsuttrykk
\cite[s.~65--67]{reitan2008.kommunikasjon}.

Kommunikasjon er mer enn bare utveksling av informasjon. Det er grunnleggende
for utøvelse av god sykepleie at man har en god samhandling med pasienten.
Travelbee sier at kommunikasjon er en prosess i stadig bevegelse, som kan være
et hjelpemiddel i sykepleiesituasjoner. Det er først og fremst et hjelpemiddel
som brukes for å bli kjent med pasienten. Det hjelper oss til å forstå og møte
pasientens behov for hjelp til mestring av sykdom, lidelse og ensomhet. I følge
Travelbee er gode ferdigheter innen kommunikasjon noe som kan læres. Det krever
at sykepleier har en intellektuell tilnærmingsmåte til problemene og bruker seg
selv terapeutisk (referert i \citeNP[s.~65]{reitan2008.kommunikasjon}). Som
sykepleier skal man ha vilje og engasjement til å møte pasientens behov og
problemer, men det må også være rom for at man kan erkjenne egen usikkerhet og
ubehag ved å gå inn i vanskelige samtaler
\cite[s.~81]{reitan2008.kommunikasjon}.

Når det kommer til humor er det ingen faste regler å forholde seg til.
Pasienter er forskjellige og reagerer på ulike måter. Man må alltid se an
personen man er sammen med, og situasjonen man befinner seg i. Det passer seg
ikke alltid med humor før man er blitt godt kjent med pasienten (Arnold \&{}
Boggs i \citeNP[s.~247]{eide2008}). Noe som også bekreftes av
\citeA[s.~294]{bjork2011}. I følge dem er humor noe individuelt og personlig
som må brukes med forsiktighet.

Vennligsinnet humor kan være god medisin. Det kan være viktig for opplevelsen
av livsglede, omsorg og fellesskap. Humor og latter har alltid vært en del av
menneskets sosiale relasjoner. Dersom det brukes med forstand, er det en
verdifull egenskap. Det hevdes at mennesker med en positiv innstilling er
friskere og lever lenger, i forhold til personer med en negativ innstilling
\cite[s.~55]{bohn2000}. Det positive med humor er at det kan løsne på
stemningen, redusere vanskelige følelser, tanker og styrke kontakten mellom
sykepleier og pasient (Arnold \&{} Boggs i \citeNP[s.~247]{eide2008}). Som
sykepleier er det lurt å lytte til pasienten. La pasienten bruke humor om sin
egen situasjon, men være tilbakeholden med egne humoristiske bemerkninger.
Dette gjelder særlig før man kjenner pasienten godt. Humor, latter, lykke og
glede kan observeres hos mennesker i en stor krise. Sykepleiere bør legge vekt
på å stimulere disse positive uttrykkene i sitt arbeid \cite[s.~169]{wist2002}.
I følge Arnold og Boggs og Burnard kan humor være en nyttig og virkningsfull
kommunikasjonsstrategi (referert i \citeNP[s.~244]{eide2008}).

Mechanic uttrykker at en negativ side ved humorbruk er at det kan
oppfattes som kunstig. Det kan bli brukt som en metode for å unngå seriøse
samtaler fra både sykepleier og pasient (referert i \citeNP[s.~247]{eide2008}).
I profesjonelle relasjoner er det ikke alle former for humor som er
akseptabelt; vitsing, fleip og moro er ikke alltid på sin plass. Fleip og humor
kan brukes av pasienten for å dekke over indre problemer, usikkerheter og
spenninger. Av noen brukes det som en form for kontroll. Flere pasienter bruker
galgenhumor i forhold til situasjonen de befinner seg i, dette kan hjelpe dem
til å distansere seg fra sorg og smerte i en viss periode
\cite[s.~56]{bohn2000}. Humor mellom pasient og sykepleier skal være noe som
kan deles, man skal le sammen og ikke av \cite[s.~166]{wist2002}.

I følge Lang mfl.~kan stressnivået ofte være høyt blant personalet. Man vil
kunne ha behov for å få utløp for sine tanker sammen med kolleger, gjerne ved
bruk av fleip og humor. Det kan ofte bli brukt galgenhumor i denne sammenheng.
Men det er viktig at denne type humor blir holdt internt blant personalet, da
den kan oppleves nedverdigende for pasientene (referert i
\citeNP[s.~245]{eide2008}).

Det er flere former for humor som kan oppleves krenkende. Eksempler på slike
kan være sarkasme, ironi og fleip. Det er viktig å ha en forsiktighet rundt
dette og være sikker på at pasienten setter pris på det før man bruker denne
type humor \cite[s.~246]{eide2008}.

\section{Humor som relasjonsbygging}

I sykepleiepraksis er relasjonen mellom pasient og sykepleier vesentlig. To
mennesker med sine styrker og svakheter møtes \cite[s.~889]{eriksen2015}. For å
bygge en menneske til menneske relasjon er sykepleiers holdninger,
forventninger og åpenhet for informasjon av betydning
\cite[s.~89]{brataas2011}. Man må som sykepleier forholde seg til pasientens
erfaringer, med deres meninger, uttrykk og følelser.
Relasjonen til sykepleier er viktig for pasientens opplevelse av pleien. For at
sykepleier skal kunne ivareta pasientens psykososiale behov, må man ta
utgangspunkt i den enkelte pasient.  Man må bruke god tid til å bli kjent med
pasienten for å kunne avdekke hans behov \cite[s.~889--901]{eriksen2015}.

I Travelbee sin definisjon av sykepleie sier hun at
\textquote[{\citeNP[s.~29]{travelbee2001}}]{sykepleie er en mellommenneskelig
prosess der den profesjonelle sykepleiepraktikeren hjelper et individ, en
familie eller et samfunn med å forebygge eller mestre erfaringer med sykdom og
lidelse og om nødvendig å finne mening i disse erfaringene}. Videre forklarer
hun at det er en \textquote{mellommenneskelig prosess} fordi det alltid dreier
seg om mennesker, enten direkte eller indirekte.

Når humor blir brukt for å bygge relasjoner, kan den være kontaktskapende
\cite[s.~192]{spurkeland2002}. Arnold og Boggs skriver at det kan bidra til økt
nærhet og det kan hjelpe med å styrke båndet mellom pasient og sykepleier,
dersom det allerede er etablert en god og trygg kontakt (referert i
\citeNP[s.~247]{eide2008}).

For en pasient kan det å motta dårlige nyheter føre til psykisk smerte. I følge
Kari Martinsen har omsorg med relasjoner og moral å gjøre, dette viser seg
igjennom praktisk handling. For at sykepleier skal kunne hjelpe pasienten, må
man  først og fremst kunne sette seg inn i deres situasjon (referert i
\citeNP[s.~80--81]{reitan2008.kommunikasjon}).

Humor kan ha stor positiv effekt, men det kan også brukes på en negativ måte
for eksempel ved å latterliggjøre eller kritisere. Dette bør ikke forekomme i
relasjoner til pasient. Spøk og fleip kan ha et skjult, negativt budskap
\cite[s.~246]{eide2008}. Om en situasjon oppleves som god eller dårlig av
pasienten avhenger av sykepleierens væremåte og holdning.  Sykepleiers
utfordring blir å møte pasienten på en god måte, hvor moral og etikk viser seg
gjennom væremåte, kroppsspråk og det som blir sagt og gjort
\cite[s.~127]{brinchmann2008}.

Å hjelpe et menneske til å mestre sykdom kan være vanskelig, å desto
vanskeligere dersom fremtiden er usikker. Her er det viktig med gode relasjoner
og kunnskap om hva den enkelte trenger for å finne mening når livet er
vanskelig. Travelbee sier at det er først når et godt samarbeid mellom pasient
og sykepleier er oppnådd at man kan hjelpe den syke med å finne håp og mening,
og hjelpe dem til å mestre sykdom og lidelse. Videre sier hun at en sykepleier
må strebe etter forandring, slik at pasientens helse opprettholdes best mulig
\cite[s.~30]{travelbee2001}.

\section{Humor og mestring}

Lazarus og Folkman har utarbeidet en definisjon som sier at mestring er
\textquote[{sitert i \citeNP[s.~65]{heggen2010}}]{stadig skiftende kognitive og
handlingsrettede forsøk som tar sikte på å handtere spesifikke ytre og/eller
indre utfordringer som blir oppfattet som byrdefulle eller som går utover de
ressursene som personen rår over}. Dette er en prosessorientert definisjon hvor
det blir lagt vekt på hva mennesket tenker og gjør, ikke hvordan mennesket er.
Den er knyttet opp til en bestemt utfordring, og skiller mellom handlingen og
utfallet. Mestring er i denne sammenheng en målrettet handling, som kan være
både vellykket og mindre vellykket \cite[s.~65]{heggen2010}. I følge Gjærum kan
ikke mestring sammenliknes med å fikse situasjonen. Det blir begrunnet med at
enkelte situasjoner ikke har noen eksakte løsninger, men mestringsmuligheter
(referert i \citeNP[s.~57]{reitan2008.kommunikasjon}).

Mestring handler om hvordan mennesker håndterer belastende livssituasjoner. Om
man ser enkelt på det, kan man si at det er to måter å reagere på. Man kan
enten mobilisere indre og ytre ressurser for å møte utfordringen, eller man kan
bruke ulike forsvarsmekanismer for å fortrenge eller unngå ubehaget
\cite[s.~157]{reitan2006}. Man må ta i bruk de ressurser man har for å bedre
situasjonen. Enten det er for å takle stress, påkjenninger, kriser eller
sykdom, slik at man kan komme seg videre på best mulig måte
\cite[s.~64]{heggen2010}.

Det er flere faktorer som påvirker menneskets evne til mestring. Noen eksempler
på dette kan være kjønn, alder og personlige interesser. Tidligere erfaringer
med sykdom, stress eller krise har også betydning for evnen til å mestre
\cite[s.~58]{reitan2008.kommunikasjon}.

Mennesker er humoristiske individ \cite[s.~242]{eide2008}. Borge og
Kristoffersen viser til en studie av eldre, nyopererte pasienter som viser at
pasientene syntes at sykepleierne brukte for lite humor. I de tilfellene hvor
humor ble brukt, var det pasientene selv som tok initiativet. Humor kan være
viktig for pasientenes trivsel og det kan brukes som en mestringsstrategi, både
for pasient og sykepleier (referert i \citeNP[s.~242--243]{eide2008}).

Mechanic skriver at humor kan være en god mestringsstrategi (referert
\citeNP[s.~244]{eide2008}). Man kan aktivt bruke humor for å mestre situasjonen
man befinner seg i. Det kan brukes for å beskytte seg selv i en tøff hverdag,
men samtidig bør det ikke brukes for å fornekte situasjonen. Man kan ofte
observere at pasienter som spøker med alvoret, reduserer spenningen i
situasjonen \cite[s.~164]{wist2002}.

Ved alvorlig sykdom kan man bli fokusert på egne problemer og lidelser, dette
medfører at evnen til å forholde seg åpent til andre kan være redusert. Humor
kan i denne sammenheng åpne opp, lette på stemningen og skape kontakt. Det kan
også gjøre det lettere å snakke om det som er vanskelig. Pasienten kan befinne
seg i en tragisk, fortvilende eller absurd situasjon, hvor god humor kan føre
til forsoning med dette. Humor kan gjøre det mulig å være nær det som er
vanskelig, om så bare for en liten periode \cite[s.~244--245]{eide2008}. Lang
mfl.~uttrykker at humor kan virke befriende fra den vanskelige situasjonen, ved
at det skaper en avstand fra alvoret (referert i \citeNP[s.~245]{eide2008}).
