\chapter{Innledning}

Kreft er en sykdom som rammer mange mennesker. Det er en alvorlig og
livstruende sykdom som innebærer mange utfordringer og følelser som angst, uro
og bekymringer for egen livssituasjon. En slik omveltende forandring vil kunne
påvirke livskvaliteten og livet generelt. Flere opplever at de stenger omverden
ute, eller at andre trekker seg unna når de har fått en kreftdiagnose
\cite{rustoen2008}.

I dag lever man lenger med kreft enn hva man gjorde tidligere. Dette medfører
at pasientene har andre behov og at helsepersonell har flere oppgaver. Det
handler ikke bare om pasientens fysiske helse, men også hvordan man kan øke
pasientens livskvalitet ved å gjøre livet til den enkelte bedre
\cite{rustoen2008}.

Kreft er ikke en sykdom som forbindes med noe komisk. Det forventes at man skal
gråte, ikke le. Selv om man får en kreftdiagnose, betyr det nødvendigvis ikke
at man mister sin humoristiske sans. Livssituasjonen og behovene kan endres,
men man er fremdeles det samme mennesket. Bruk av humor i kreftomsorgen kan ha
viktig betydning for pasienten, det kan hjelpe til å bevare følelsen av å være
et levende menneske \cite{wist2002}.

\section{Bakgrunn for valg av tema}

Etter hjerte- og karsykdommer er kreft den vanligste dødsårsaken i Norge. Hvert
år dør i overkant av 10 000 mennesker av kreft. I 2014 fikk 31 651 nordmenn en
kreftdiagnose og 10 971 døde. Risikoen for å få kreft øker med alderen, men
rammer alle aldersgrupper \cite{grasdal2016}. Stavanger Aftenblad skriver at
Rogaland er et av de fylkene i Norge som er hardest rammet av kreft. Man regner
med at så mange som én av tre vil få en kreftdiagnose før fylte 75 år.
Hovedgrunnen til dette er at vi i dag lever lenger \cite{haugan2015}.

Som sykepleierstudenter og fra tidligere erfaring har vi merket oss at humor
kan ha stor positiv effekt på samarbeidet med pasienter. Vi vet at humor kan
være kontaktskapende når den brukes til å bygge relasjoner. Vi opplever at det
er med på å åpne opp for et godt samarbeid, og å skape en god relasjon mellom
sykepleier og pasient. Vi har også erfart at man gjerne blir litt mer
tilbakeholden til bruk av humor ved alvorlig sykdom, og gjerne ekstra ved den
siste fasen av livet. Vi ønsket derfor å se litt nærmere på hvordan humor kan
brukes i kreftomsorgen, og hvilken effekt det har i møte med pasienter som
lider av kreftsykdom.

\section{Problemformulering}

Kreft er en alvorlig og livstruende sykdom. Mange sykepleiere vegrer seg derfor
mot å bruke humor i samhandling med kreftpasienter. Flere er usikre på humorens
virkning på pasienter, eller er <<redde>> for at humoren kan bli misforstått og
få motsatt effekt. Vi ønsker å finne ut hvordan humor blir mottatt av pasienter
med kreft. Det er viktig at vi som sykepleiere vet hvordan vi skal bruke humor
i kommunikasjon med pasienter, og hvilken virkning humoren kan ha i samhandling
med kreftsyke pasienter. Samt at vi vet noe om fordeler og ulemper ved bruk av
humor.

Det er lite litteratur og forskning på bruk av humor i sykepleie, så det er
vanskelig å vite noe om hvordan det fungerer. Man kan oppleve at det blir tatt
i bruk på feil måte, eller at man er redd for å bruke humor. For å vite hvordan
man kan bruke humor, må man vite noe om hva pasienter opplever som positivt og
negativt ved bruk av humor.

\section{Hensikt}

Oppgavens faglige hensikt er først og fremst å se på hvordan humor blir
praktisert i sykepleien. Vi ønsker å finne ut hvordan humor kan brukes i
relasjon mellom pasient og sykepleier. I tillegg ønsker vi å ha særlig fokus på
humor i kommunikasjon og som mestringsstrategi

\section{Begrepsavklaring}

\textbf{Humor:} Vi har alle ulike oppfatninger om hva humor er, det er
derfor vanskelig å gi en presis definisjon. Vi har sett på ulike definisjoner
og fant blant annet en som beskriver det slik:
\blockquote[{\citeNP[s.~15]{tyrdal2002}}]{Den mentale evne til å oppdage,
uttrykke eller sette pris på elementer av latterlig eller absurd inkongruens
(motstridene) i idéer, situasjoner, hendelser eller handlinger}.  I vår oppgave
har vi valgt å forenkle dette ved å si at \textit{humor er evnen til å oppfatte
noe som er morsomt}.

\section{Avgrensning av oppgaven}

Vi har valgt å ha fokus på sykepleier, pasient, og samhandlingen dem imellom.
Vi vet at pårørende er en viktig ressurs for den kreftrammede og for oss som
helsepersonell, allikevel har vi valgt å ikke legge vekt på dette i vår
oppgave. Mye av teorien har vi funnet fra \citeA{rustoen2008},
\citeA{eide2008}, \citeA{travelbee2001} og \citeA{tyrdal2002} sine bøker. Dette
er bøker som gir oss sykepleier- og pasientperspektiv. Pasientene i oppgaven er
voksne personer med en kreftsykdom.  Selv om oppgaven omhandler kreftpasienter,
er det ikke sykdommen som er vårt hovedfokus.
