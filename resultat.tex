\chapter{Resultat}

\section{Bruk av humor i kommunikasjon}

\subsection{Relasjonsbygging mellom sykepleier og pasient}

Artikkelen til Tanay el al. (2013: 1297) viste at både sykepleier og pasient så
på humor som et tegn på at det ble utviklet et personlig forhold mellom dem.
Flere pasienter opplevde at de hadde et godt forhold til sykepleier dersom de
lo sammen. Sykepleierne oppfattet viktigheten humor hadde for å etablere en god
relasjon til pasientene, og effekten det hadde for å få pasientene til å føle
seg mer avslappet. Ved bruk av humor fikk sykepleierne vise at de også var
mennesker, og dette gav en følelse av tilhørighet for pasienten og de rundt
dem. Det virket som om pasientene fikk større tillit til sykepleierne dersom
det ble brukt humor. På den måten ble det lettere for pasientene å åpne seg, og
snakke om de vanskelige tingene (Tanay, Wiseman, Roberts \&{} Ream, 2013: 1297).

I studien til Tanay et al. (2013: 1298) så det ut til at sykepleiere foretrakk
morsomme pasienter. Det var lettere for pasienten å ta kontakt med en
sykepleier som viste en form for humoristisk sans, i forhold til en som ikke
hadde det. En pasient sier at man ved bruk av humor hadde større sjanse for å
bli likt. Dersom man brukte humor til å spørre om noe, svarte sykepleierne
muntrere. En pasient beskriver det slik: “Somebody with a sense of humour
asking for a cup of tea, is more likely to get one than somebody demanding a
cup of tea”. (Tanay et al.,2013: 1298)

McCreaddie og Payne (2001) finner i sin studie at humor ikke alltid var
positivt. Noen pasienter kjente på behovet for å være en “god” pasient, og
prøvde dermed å adoptere sykepleierens væremåte. Dersom sykepleieren brukte
humor, gjorde også pasienten det for å oppnå bedre kontakt og for å få den
hjelpen de trengte. De fant også at pasientene brukte humor til å uttrykke
følelser og engstelse. Dette kunne være risikabelt da sykepleier ikke oppfattet
alvoret fra pasienten.

\subsection{Den vanskelige samtalen}

Sykepleiere mente at humor fikk pasientene til å føle seg hjemme og være mer
avslappet. De tok dette som en indikasjon på at pasientene stolte mer på dem
dersom de hadde ledd litt sammen. Dette førte til at pasientene åpnet seg mer
og ønsket å snakke om de seriøse tingene (Tanay et al., 2013:1297).  Humor ble
av pasientene brukt som beskyttelse når de for eksempel gikk tom for ord, eller
for å slippe å vise sårbarhet (Roaldsen et al., 2015: 4). Å sette ord på
erfaringer kunne være vanskelig. Det å bruke humoristiske uttrykk kunne
indirekte være en måte å kommunisere forståelig med andre. De brukte
humoristiske metaforer og bilder med varierende intensjoner, som for eksempel å
stanse sensitive temaer uten at samtalepartneren skulle føle seg avvist.
Sykepleierne sa at de ved å bruke humor på den måten kunne distansere seg selv
fra pasientene, og slippe å måtte gå inn i seriøse diskusjoner med dem (Dean \&
Major, 2007).

I Roaldsen et al. (2015) reagerte en pasient med latter da hun fikk diagnosen
brystkreft. Hun sa at dette var den eneste måten hun greide å reagere på, og
sa: “ You mustn’t laugh, because then they’ll think you’re crazy, ‘cause people
are supposed to cry” (Roaldsen et al., 2015: 4).  Innen palliativ pleie var det
ofte humor involvert når de snakket om fortiden. Mimring om fortiden var
spesielt viktig for pasientene da de ble konfrontert med at det gikk mot
slutten. Av og til delte de byrder fra fortiden, men ofte også høydepunkter
gjennom livet. Dette var også en mulighet for personalet å få god kontakt med
pasienten og deres pårørende (Dean \&{} Gregory, 2005).

\subsection{Individuelle hensyn ved bruk av humor}

I studien til Dean \&{} Gregory (2005) kom det frem flere faktorer som var
avgjørende for bruk av humor i sykepleien. I tillegg til etnisitet, kjønn og
stressnivå  var bruk av humor avhengig av pasientens personlighet og
situasjonen pasienten var i. I tillegg uttrykte noen sykepleiere en utrygghet
ved bruk av humor i jobben. De fryktet at det skulle gå utover deres
profesjonalitet og de var bekymret for hvordan deres kolleger skulle se på dem.
Dette kan være grunnen til at mange unge sykepleiere var mer seriøse på jobb i
møte med eldre sykepleiere. Når det bare var unge sykepleiere ble det observert
mer humor og tøys (Tanay et al., 2013:1298).

Det ble pekt på flere omstendigheter hvor deltakerne mente at humor ikke var
passende. Ved endring i pasientens tilstand kunne det være mye frykt, sinne og
sorg. I slike situasjoner ble forsøk på bruk av humor ikke satt pris på av
pasientene. Det kom frem flere situasjoner hvor pleierne forsto at de hadde
gått for langt, og at humor ikke var passende (Dean \&{} Gregory, 2005). I
McCreaddie \&{} Payne (2011) var det derimot enighet blant pasientene om at de
satte pris på sykepleiernes bruk av humor også i situasjoner hvor man ikke
skulle tro det var passende. Pasientene syntes det var positivt dersom
sykepleierne var muntre og glade. Det kunne bidra til å lette på stemningen i
vanskelige situasjoner. Pasientene syntes også at det var bra at sykepleierne
tok noen sjanser, og at de ikke var for redde for å bruke humor.

I situasjoner hvor pasienten lå på dødsleiet var det av den oppfatning at
humoren skulle bli overlatt til pasienten og pårørende. Det var her ikke
passende for personalet og komme med kommentarer av humoristisk karakter (Dean
\&{} Gregory, 2005). En sykepleier fortalte om rørende øyeblikk rett før døden.  “
in their last minutes of life I’v seen humour used there too. It’s a very
loving humour, it’s kind of heart-to-heart humour from a family member to the
one who’s dying” (Dean \&{} Gregory, 2005: 296).

Både pasient og sykepleier mente at det var viktig å ta hensyn til etnisitet.
Det var likevel vanskelig for deltakerne i undersøkelsen til Dean \&{} Gregory
(2005) å gi noen spesifikke eksempler på hensyn man måtte ta, utenom at det var
et behov for sensitivitet og forsiktighet rundt bruk av humor. En pasient mente
at humor var vanskelig mellom mennesker fra forskjellige kulturer. Språket
gjorde at det kunne oppstå misforståelser som følge av feil uttalelser eller at
ting ble feiltolket.  Det var forskjell i hva kvinner og menn foretrakk ved
bruk av humor. Menn hadde en tendens til å bruke humor som et middel for å være
åpne med hverandre og for å dekke over ubehag. Deres bruk av humor var også mer
preget av seksuelle bemerkninger over for personal, noe som gjorde at mange
kvinnelige sykepleiere følte seg utilpass. Noen håndterte dette ved å overse
kommentarene, mens andre vitset det bort (Dean \&{} Gregory, 2005).

I Dean \&{} Gregory (2005) hadde deltakerne vanskelig for å svare på når og
hvordan de brukte humor. Mange svarte at det ikke var noe bevisst, men at det
bare oppsto spontant. Andre igjen sa at det var vel overveid og at timingen
måtte være rett. I Studien til Tanay et al. (2013) sa noen at de hadde en
intuisjon om når det passet. Humor er individuelt og det er forskjeller i
hvordan det blir mottatt. For å ikke ødelegge sykepleier - pasientrelasjonen
var det viktig at man vurderte situasjonen og så etter tegn fra den andre
personen før man brukte humor.

\section{Mestring}

\subsection{Bruk av humor som mestringsstrategi}

I artikkelen til Roaldsen et al. (2015) snakket noen pasienter om humor som kom
og gikk. I perioder av sykdommen som var preget av mye uvisshet, angst og
stress kunne sansen for humor forsvinne. I bedre perioder kunne humoren komme
tilbake. Man kunne da ofte se at typen og bruken av humor var noe endret. De
brukte for eksempel mer galgenhumor og ironi rettet mot sin egen situasjon, og
sine erfaringer med kreftsykdom. Denne typen humor ble gjerne brukt som en
slags beskyttelse for å dekke over hvor vanskelig de egentlig hadde det. En
pasient beskriver det slik: “That’s how you blossom. Humour keeps away those
heavy thouhts. Yes, it gets more like gallows humour. You use it, quite
obviously, to put thing at a distance” (Roaldsen et al., 2015: 5).

En annen pasient sa at galgenhumor var en måte å beskytte seg selv ved å dekke
over det hun virkelig følte, som var hvor fæl denne erfaringen hadde vært. “...
if I couldn`t have laughed and had fun with it, well I think I’d have had to go
through a very dark time, I mean, I don`t think I`d have managed to be so
strong” (Roaldsen et al., 2015: 5).

I samme artikkel sa andre at i perioder med kaos og fortvilelse, var humor
særlig viktig for å håndtere følelsene. På den måten overtok ikke sykdommen all
plass. Humor hjalp pasientene til å distansere seg fra negative følelser og
tunge tanker. Dette kunne de gjøre ved å se morsomme tv program, eller
underholdende klipp på YouTube. Dette lot dem få avstand fra ensomheten og
tankene om døden. Humor og latter var viktig både for helsepersonell og
pasientene for å håndtere spenningen og tristheten som ofte dukket opp. Dette
skapte pusterom fra de tunge situasjonene de befant seg i.

Under kreftbehandling fortalte pasientene at deres mestring av situasjonen ble
satt på prøve. De var derfor bevisste sine mestringsstrategier og understreket
at humor var viktig for å skape en levelig situasjon. Muligheten til å
akseptere og tåle sykdom var knyttet til et humoristisk livssyn som gav
opplevelse av mening, og det å være deltakende i samfunnet. Det betydde at man
måtte takle motgang og akseptere sykdommen som et faktum. Pasienter sa at det
var viktig i denne sammenheng å finne en meningsfull balanse der sykdommen ikke
ble for dominerende (Roaldsen et al., 2015).

Flere deltakere foralte at deres bruk av humor var påvirket av graden av stress
de erfarte. Det interessante her var at enkelte brukte mer humor jo høyere
stress de erfarte. I kontrast, var andre mindre mottakelige for bruk av humor
ved mye stress (Dean \&{} Gregory, 2005).

Humor kunne gi et lysere perspektiv på en vanskelig situasjon. Ved å overføre
en tragisk situasjon til underholdende ord, kunne man le av fortvilelse, sinne
og sorg. En pasient i Roaldsen et al. (2015) bestemte seg for å leve i
øyeblikket, da han fikk vite at det var fare for spredning av sykdommen.

\textquote[side 6]{
 [...] then I sat for a long time thinking: What if I die? The kids. What about
 them? My wife, the money, the house? But suddenly I thought: No, bloody hell!
 There is another alternative, and that's that everyting's fine! You can't bury
 yourself in seriousness, then you might as well close the lid [...]
}

Videre sa han at humor kanskje ikke hjelper deg til å overleve, men at det kan
gjøre livet bedre oppi alt annet.

I palliativ pleie, hvor døden kunne være nær, hadde følelser en tendens til å
være forsterket. Humor kunne da maskere underliggende følelser. Dette kunne
sende ut feil signaler og føre til at pasientene ikke fikk den hjelpen de
trengte. De erfarne sykepleiere som jobbet rundt disse pasientene, lærte seg
etterhvert å se hva som egentlig  gjemte seg bak humoren (McCreaddie \&{} Payne,
2011).
