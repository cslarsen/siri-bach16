Det å få en kreftdiagnose skaper store omveltninger i livssituasjonen og kan
medføre mye sorg og fortvilelse. Vi har selv erfart at humor kan brukes selv i
de vanskeligste situasjoner, og kan hjelpe pasienten gjennom tøffe tak.
Hensikten med vår oppgave var å se nærmere på hvordan sykepleier bruker humor i
arbeidet sitt og hvordan pasientene selv bruker humor i hverdagen som kreftsyk.
Vi har valgt å gjøre en litteraturstudie hvor vi har gjort en analyse av
kvalitative studier, i tillegg til bruk av litteratur og egne erfaringer.
Resultatene vi fant var at humor spiller en vesentlig rolle i omsorgsarbeidet.
Man ser at humor er kontaktskapende og kan brukes for å bygge relasjoner mellom
sykepleier og pasient. Dette fører til et økt tillitsforhold som igjen bidrar
til et bedre samarbeid og at pasientene får den behandlingen de trenger. Vi
fant også at pasientene bruker humor som en mestringsstrategi for å bedre
håndtere og klare å leve med sykdommen.
