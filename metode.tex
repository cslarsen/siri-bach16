\chapter{Metode}

Vi har valgt å gjøre en litteraturstudie med kvalitativ metode som
utgangspunkt. En litteraturstudie er en studie hvor innsamlingsdata hentes fra
litteraturen, og man får en strukturert oversikt over et valgt tema (Segesten,
2006). Kvalitativ metode er en forskningsmetode som gir beskrivende data. Den
søker å gi en dypere forståelse og en økt kunnskap om et tema. Ved hjelp av
intervju og/eller observasjon får man innsikt i menneskers personlige
opplevelser og erfaringer (Olsson \&{} Sörensen, 2003: Sidetall). Vi har valgt å
analysere fem kvalitative forskningsartikler etter Fribergs modell i “Dags för
uppsats” (2006). Kvalitative studier gir en økt forståelse for hvordan man kan
møte pasientenes behov gjennom å se på pasientens opplevelser, erfaringer og
forventninger (Segesten, 2006: sidetal). Dermed er kvalitativ forskning den
metoden som er best egnet for besvarelse av vår problemstilling.  Vi ønsker å
se nærmere på hvordan man kan bruke humor i arbeid med kreftpasienter, hvordan
det blir mottatt, og vi ønsker å vite noe om hva pasientene opplever som
positivt og negativ ved bruk av humor. Gjennom grundig analyse av artiklene
finner vi gode beskrivelser av pasientenes og helsepersonells egne erfaringer,
opplevelser og atferd, gjort gjennom observasjoner og intervju.

\section{Litteratursøk}

Litteratursøket til oppgaven ble gjennomført i perioden 26.11.2015 til
03.12.2015. Databasene vi brukte til å søke opp artikler var Cinahl og Oria. Vi
har også brukt tidsskriftet Scandinavian Journal of Caring Sciences’ egen
søkemotor for vitenskapelige artikler. De fleste artiklene fant vi i databasen
Cinahl. Dette er en database som blir mye brukt til søk av vitenskapelige
artikler innen helsefag, og var derfor godt egnet for oss.  I Cinahl valgte vi
å begrense søket til å gjelde fra 2005 til 2015 slik at vi skulle få de senest
publiserte artiklene, og den nyeste kunnskapen. Vi huket av på ”full text” og
“peer reviewed” da dette indikerer at artiklene er vitenskapelige og
kvalitetssikret av andre. Søkeordene som ble brukt var engelske. Dette fordi de
fleste vitenskapelige publikasjoner utgis på dette språket. Grunnen til det er
at forfatteren vil nå ut til flest mulig med sin forskning (Friberg, 2006:
sidetall). Søkeord som Humor, Humour, Health care og Nurs ble brukt. Vi fant
fire artikler her som vi ønsket å undersøke videre.

I Oria huket vi av for “artikler” og brukte søkeordene “Humor” og
“Relationship”. Her fikk vi 1464 treff. Videre avgrenset vi for årstall i
tidsrommet 2005 - 2015, og at artiklene skulle være fagfellevurdert. Dette
førte til 584 treff. Blant dem fant vi én som vi ønsket å se nærmere på.
Artikkelen var tilgjengelig i fulltekst hos ProQuest Health and Medical
Complete.

Den siste artikkelen vi ønsket å se nærmere på fant vi i Scandinavian Journal
of Caring Sciences. Her ble søkeordene “humour” og “cancer” brukt. De fleste
artiklene ble forkastet da de ikke oppfylte våre kriterier, og av 17 treff var
bare én av disse relevant for oppgaven vår.

\subsection{Oversikt av analyserte artikler}

Se tabell \vref{tabell.artikler}.

\begin{landscape}
  \begin{table}
    \centering
    \small
    \begin{tabularx}{\paperwidth}{
        >{\raggedright\arraybackslash}X
        >{\raggedright\arraybackslash}X
        l
        >{\raggedright\arraybackslash}X
        >{\raggedright\arraybackslash}X
        l}
      \textbf{Artikkel} &
      \textbf{Forfattere} &
      \textbf{Tilnærming} &
      \textbf{Metode} &
      \textbf{Deltakere} &
      \textbf{Søkemetode} \\
      \\
      Cancer survivors’ experiences of humour (2015) &
      Roaldsen, B.L.; S{\o}rlie, T.; Lorem, G.F. &
      Kvalitativ studie &
      Intervju i perioden 2010--2011 av første forfatter &
      14 pasienter i alderen 23--83 år; 7 kvinner og 7 menn &
      Søk i tidsskrift
      \\ \\
      A time to weep and a time to laugh (2013) &
      Tanay M.A.; Wiseman, T.; Roberts J., Ream, E. &
      Kvalitativ studie &
      Intervju, semistrukturerte intervju, uformelle intervju og observasjon &
      9 sykepleiere og 12 pasienter ble observert. 5 sykepleiere og 5 pasienter ble intervjuet. &
      Søk i database
      \\ \\
      More than trivial (2005) &
      Dean R.A.; Gregogy D.M. &
      Kvalitativ studie &
      Observasjon av sykepleiere over seks uker. Intervju med
      pasienter og pårørende. Semistrukturerte intervju med helsepersonell. &
      6 sykepleiere ble observert; 11 sykepleiere, 2
      sosialarbeidere og 1 fysioterapeut ble intervjuet &
      Søk i database
      \\ \\
      Humour in health-care interactions (2011) &
      McCreaddie, M.; Payne, S. &
      Kvalitativ studie &
      Intervju, observasjon, notater og lyddagbok &
      32 deltakere, 4 pasientfokusgrupper &
      Søk i database
      \\ \\
      From critical care to comfort care (2007) &
      Dean, R.A.K.; Major, J.E. &
      Kvalitativ studie &
      Sammenligning av to tidligere studier &
      &
      Søk i database
    \end{tabularx}
    \label{tabell.artikler}
    \caption{Oversikt av analyserte artikler}
  \end{table}
\end{landscape}


\section{Analyse}

Analyse av tekstene har vi gjort med utgangspunkt i Fribergs analysemodell i
“Dags för uppsats” (2006). Friberg mener det er viktig å fokusere på studienes
resultat, og lese gjennom artiklene flere ganger slik at man får ordentlig tak
på hva de handler om. På denne måten blir det enklere å plukke ut hovedfunn og
gjøre en sammenligning mellom de ulike artiklene. Friberg benytter en
hermeneutisk tilnærming i analysearbeidet, og beskriver det som en bevegelse.
Når man har valgt ut sine artikler, deler man artiklenes resultater inn i de
kategorier man søker. Deretter setter man det sammen igjen til et nytt
resultat. Her ser man at arbeidet går fra en helhet, til deler og tilbake til
en ny helhet (Friberg, 2006: sidetall).

Etter å ha lest gjennom abstraktet, valgte vi ut de artiklene vi syntes var
mest relevante. Det var vanskelig å finne artikler som omhandlet både kreft og
humor, og som samtidig holdt opp til våre kriterier. Derfor valgte vi å bruke
noen artikler som omhandlet palliasjon, da dette kan være en naturlig del av et
kreftforløp. Med hovedfokus på tema og resultat, leste vi gjennom alle
artiklene for å få en oversikt over hva de handlet om, og om de oppfulgte våre
kriterier. Noen av artiklene ble byttet ut underveis, da vi ikke syntes de ga
svar på det vi søkte etter. Deretter finleste vi artiklene for å finne likheter
og ulikheter mellom de forskjellige. Gjennom fargekoder og skjematisk
fremstilling sorterte vi de ulike funnene. På den måten fikk vi en god oversikt
over likheter og ulikheter i artiklene. Ut i fra dette kunne vi dele det videre
opp i hoved- og underkategorier. Dette var et krevende arbeid. Det var mange
interessante resultater, og mye vi kunne tenkt oss ha med, så det var
utfordrende å holde fokus på det vi søkte etter. På bakgrunn av funnene vi
gjorde, satte vi til slutt sammen to hovedtema med tilhørende  subtema.

\todo{hva skal dette være}
Bruk av humor i kommunikasjon

	* Relasjonsbygging mellom sykepleier og pasient
* Humorens motsatte effekt
* Den vanskelige samtalen
* Individuelle hensyn
	Bruk av humor som mestringsstrategi

	*  Håndtering av følelser
*  Tilbakevennende humor


