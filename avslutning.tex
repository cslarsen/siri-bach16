\chapter{Avslutning}

Humor har alltid blitt omtalt som livets medisin, og en viktig helsefaktor. Det
har blitt gjort noen få vitenskapelige undersøkelser vedrørende humorens effekt
i arbeidet med syke, men det man ser er at bruk av humor berører flere sider
ved et menneske. Vi ønsket å se nærmere på hvordan humor ble brukt i relasjon
mellom sykepleier og pasient, og hvordan pasienter bruker humor som en form for
mestringsstrategi. Vi fant at humor ble brukt som et kommunikasjonsmiddel for å
bygge relasjoner mellom sykepleier og pasient, og at begge så på bruk av humor
som et tegn på at det ble utviklet et godt forhold mellom dem. Ved å ha en god
relasjon økte pasientens tillit til sykepleier, og man fikk større åpenhet og
bedre samtaler. Noen pasienter brukte humor som en form for mestringsstrategi,
hvor humor ble brukt for å beskytte seg, eller som en måte å takle hverdagen på
når ting ble ekstra tungt. Bruk av galgenhumor, ironi og metaforer ser vi som
eksempler på det. Humor ble stort sett oppfattet som positivt både blant
sykepleierne og pasientene. Pasientene syntes det var bra med litt munterhet i
alt det triste, og så på det som positivt hvis sykepleierne våget å ta noen
sjanser. Det ble også pekt på at humor må brukes med varsomhet, og at det ikke
passer seg like godt i alle situasjoner. Sykepleierne var i noen situasjoner
litt for tilbakeholdne med bruk av humor. Både fordi de var redde for hva
medarbeidere ville tenke, at situasjonen ikke var passende og at det ikke ville
bli godt mottatt av pasientene.
