\chapter{Diskusjon}

\section{Metodediskusjon}

Vi ønsket å se nærmere på bruk av humor som kommunikasjonsmetode og som
mestringsstrategi i sykepleien. Vi ønsket å få et innblikk i hvordan pasientene
opplever bruk av humor, hvordan sykepleiere bruker humor i sitt arbeid og
hvordan pasientene bruker humor til å mestre hverdagen. Å gjøre en empirisk
undersøkelse ville være altfor tidkrevende for oss å utføre. Vi valgte derfor å
gjøre en litteraturstudie hvor det er gjort bruk av kvalitative
forskningsartikler. Kvalitativ metode gir oss et perspektiv sett innenifra, og
gir en dypere forståelse for et fenomen blant annet gjennom intervjuer og
observasjon. Man får på den måten et bedre innblikk i menneskers tanker,
opplevelser og følelser (Olsson \&{} Sörensen, 2003: sidetall). Det kan være
vanskelig å få et bilde av tanker og følelser gjennom tall og statistikk. Vi
syntes derfor at kvalitativ metode var bedre egnet enn kvantitativ metode i
denne oppgaven.

Litteraturstudier innhenter data fra allerede analysert materiale som
vitenskapelige artikler (Friberg, 2006: sidetall). Metoden har blitt kritisert
fordi den ikke kommer frem med ny forskning, at det er for lite utvalg av
materiale og at den kan bli for subjektiv. Forskerne eller forfatterne leter
etter det de ønsker å finne, dermed kan kanskje andre relevante funn kan bli
oversett (Friberg, 2006).

Det er blir gjort relativt lite studier om temaet humor i sykepleie og utvalget
av artikler som var relevante for vår problemstilling var dermed begrenset.
Bare én av de fem artiklene vi valgte ut er norsk, mens de resterende er fra
engelskspråklige land. Vi fant svært få undersøkelser som var gjort i Norge
eller Skandinavia forøvrig, men vi tenker at samfunn og kultur er såpass likt,
at artiklene likevel kan overføres til norske forhold. Vi har ikke sett på
studier hentet fra andre deler av verden, og det kan vel derfor tenkes at denne
studien blir rettet mot den vestlige verden. Man kan jo anta at man i andre
deler av verden har kulturforskjeller hvor man har et litt annet syn på bruk av
humor enn oss, noe som igjen ville gitt oss et litt annet resultat. Selv om det
var et begrenset utvalg artikler, besvarer disse likevel vår problemstilling og
vi ser at resultatene i artiklene er gjenkjennelige fra egen praksis og
erfaringer.

Artiklene er hentet fra tre ulike databaser og er fra forskjellige land.
Likevel ser vi at det er mye de samme temaene som går igjen, og at både hensikt
og resultater samsvarer ganske mye. Dersom vi hadde utvidet søket og hatt enda
flere artikler kan det hende at vi hadde fått et litt bredere resultat enn det
vi har. Da resultatene likevel virket å være såpass like, valgte vi heller å
plukke ut noen få artikler, og fokusere på noen tema som gikk igjen.

Før vi startet søket på artikler, undersøkte vi litt rundt i litteraturen
generelt for å se hva vi fant rundt temaene “humor” og “kreft”. På bakgrunn av
dette dannet vi oss et bilde av hva vi ønsket å se nærmere på, og det gav oss
en pekepinn på hvilke søkeord vi skulle bruke. Underveis i analysearbeidet
endret fokus seg noe, og tema som “humor og fysiologisk betydning” ble for
eksempel valgt bort, da vi heller ønsket å rett fokus mot kommunikasjon og
mestring.  Tre av artiklene er av nyere dato, mens de to andre er noen år
eldre. Vi valgte likevel å ta disse med da vi ser at de er like dagsaktuelle
som de andre.  I fire av artiklene er datainnsamling hentet inn via intervju og
observasjon, mens den siste artikkelen er basert på funn fra sammenligning av
to vitenskapelige artikler. Det er gjort både semistrukturerte og uformelle
intervju. I den ene artikkelen er det kun hentet inn data via intervju, ikke
observasjon. I samtlige artikler hvor data er hentet inn via intervju og
observasjon er informantene helsearbeidere og pasienter. I den ene studien er
det også pårørende som informanter, men disse har vi midlertidig ikke valgt å
legge vekt på da vårt fokus ligger på pasient og sykepleier.

I kvalitativ metode skal man kunne gå i dybden og få en nærhet til det man
ønsker å undersøke. Det er da en forutsetning at man har et mindre antall
informanter (Olsson \&{} Sörensen, 2003: sidetall). I artiklene vi har valgt
gjenspeiles dette ved at  det hovedsakelig er små grupper på mellom 20--30
informanter. Ved et lite antall informanter er det enklere for forskeren å få
en relasjon til vedkommende og på den måten få et bedre innblikk fra
informantens perspektiv.

\section{Resultatdiskusjon}

Vi vil besvare problemstillingen vår ved å sette resultatene opp mot teori og
egne synspunkter. Vi har delt drøftingen i to. I første del ser vi på hvordan
humor i kommunikasjon brukes i relasjonsbygging og i den vanskelige samtale,
samt hvilke hensyn som bør tas ved bruk av humor. I del to om mestring ser vi
på hvordan pasienten kan bruke humor som en mestringsstrategi for å håndtere
hverdagen som kreftsyk.

\subsection{Bruk av humor i Kommunikasjon}

\subsubsection{Relasjonsbygging mellom sykepleier og pasient}

Humor kan i følge Arnold \&{} Boggs være en nyttig og virkningsfull
kommunikasjonsstrategi. Det kan være viktig for pasientens trivsel og kan
brukes til å redusere stress og spenning (referert i Eide \&{} Eide, 2008) I
sykepleiepraksis er relasjonen mellom sykepleier og pasient viktig. Vi vet at
det er mye lettere å få til et godt samarbeid med pasienten dersom man har en
god relasjon. Som sykepleier må man forholde seg til pasientens erfaringer og
de meninger, utrykk og følelser som følger med dette. Vi vet fra tidligere
erfaring at relasjon mellom sykepleier og pasient er viktig for hvordan
pasienten opplever pleien, dette blir også understrekt av Eriksen (2015).

Tanay et al. (2013) viser eksempler på at både sykepleier og pasient tar hensyn
til hverandre. En sykepleier fra studien sa: \textquote[Tanya et al., 2013:
1299]{I follow their (patients) lead ...}. Det kommer frem av denne studien at
pasientene så viktigheten av refleksjon og vurdering rundt sin humorbruk. En
pasient sa: \textquote[Tanya et al., 2013: 1299]{so I'd perhaps wait a bit and
find out how she (nurse) was, let her make the next move, basically}.  Dette
kan kanskje tolkes som at begge er forsiktige med bruk av humor for de ikke har
blitt helt komfortable med hverandre og har vanskelig for å lese hva den andre
synes er akseptabelt.

I artikkelen til Tanay et al. (2013) ser vi at både pasient og sykepleier så på
bruk av humor som et tegn på at det ble utviklet et personlig bånd mellom dem,
noe som førte til at pasientene følte seg mer komfortable sammen med
sykepleier. Man ser altså at man bør ha et visst forhold til en person, for å
føle seg komfortabel nok til å bruke humor. Tyrdal (2002) og Arnold \&{} Boggs i
Eide \&{} Eide (2008) understreker dette. Humor kan skape bedre kontakt mellom
sykepleier og pasient, dersom trygg relasjon allerede er til stede.  Travelbee
(2001) sier at det er først når pasient og sykepleier har et godt samarbeid at
sykepleier kan hjelpe og ivareta pasientens behov. I artikkelen til Tanay et
al. (2013) kom det frem at sykepleierne forsto at humor kunne hjelpe dem å
skape denne relasjonen, og at det førte til at pasientene følte seg mer
avslappet. Ved å bruke humor fikk sykepleierne vise at de også bare var
mennesker.  I Tanay et al. (2013) så det ut til at sykepleiere foretrakk
morsomme pasienter, noe som kan ha negative konsekvenser for pasientene. En
pasient i denne studien sa at dersom man brukte humor, var sjansen større for å
bli likt av sykepleierne og at man fikk den hjelpen man trengte. Dette ser også
vi på som negativt. Vi mener at man som sykepleier skal være profesjonell og
ikke la personlige tanker eller fordommer påvirke den hjelpen som gis. Alle
skal få den hjelpen de trenger, uansett hvem de er. I McCreaddie \&{} Payne (2011)
kom det også frem at pasientene følte de måtte innynde seg hos sykepleierne for
å få den hjelpen de trengte.

\subsubsection{Humorens motsatte effekt}

Humor ble stort sett beskrevet som positivt av deltakerne i studien til
McCreaddie \&{} Payne (2011) fordi det kunne være en god hjelp i å lyse opp
hverdagen og håndtere stress. Som vi tidligere har sett er ikke humor en
løsning i seg selv, men den kan fungere som en beskyttelse og et kortvarig
pusterom (Reitan 2008, Wist 2002). Bruk av humor kan være tvetydig. Den kan
brukes både til å maskere underliggende følelser og for å uttrykke bekymringer.
Dette kommer frem i studien til McCreaddie \&{} Payne (2011) som problematisk
dersom pasientene sendte ut signaler om hjelp i form av humor, men som ikke ble
oppfattet slik av sykepleierne. Det kunne også være et problem den andre veien,
dersom en spøk ble tatt alvorlig av sykepleier. Grunnen til at slike
situasjoner oppstod kunne være at pasient og sykepleier hadde ulik sans for
humor. Hvordan humor ble tolket var viktig i forhold til om pasienten fikk den
hjelpen de søkte. Man ser at det i slike situasjoner er viktig at man som
sykepleier er oppmerksom og ser pasienten som en helhet. Man må prøve å lese
pasienten, ikke bare ut i fra hva som blir sagt, men også det non-verbale og
kroppslige. For å kunne gi god sykepleie er relasjonen til pasienten
avgjørende, og man ser fra eksempelet over at kjennskap til pasienten kan gjøre
det enklere å avsløre om det ligger et budskap gjemt i humoren. Travelbee
(2001) sier at gode ferdigheter innen kommunikasjon er noe som kan læres, og
kan hjelpe oss i arbeidet med å forstå og møte pasientens behov.

\subsubsection{Den vanskelige samtalen}

Vi vet fra egne erfaringer at samtaler med kreftpasienter kan være vanskelige.
Vi har også merket oss at pasienter åpner seg i større grad dersom vi alt har
etablert en god relasjon. Likevel har vi opplevd at pasienter bruker humor for
å unngå de tunge samtalene. Dette kommer også til syne i Roaldsen et al. (2015)
hvor pasientene brukte humor som en beskyttelse når de gikk tomme for ord,
eller ikke ville vise sårbarhet.

Å sette ord på erfaringer kan være vanskelig. I Dean \&{} Major (2007) kommer det
frem at pasientene brukte humoristiske utrykk som metaforer, for å kommunisere
forståelig med andre og for å unngå vanskelige samtaler. Dette ser vi et
eksempel på i Roaldsen et al. (2015) der en pasient sa at hun brukte
kallenavnet “naked rat” om seg selv da hun var hårløs. Dette brukte hun for å
beskytte seg selv. Wist (2002) beskriver at humor kan brukes for å dekke over
usikkerheter rundt egen situasjon. Vi har alle opplevd tilfeller hvor pasienter
bruker morsomme kallenavn om seg selv. I slike tilfeller er det ofte
usikkerheten på seg selv som ligger bak. I følge Mechanic kan en slik humor
være negativ fordi det kan oppfattes som kunstig munterhet for å unngå seriøse
samtaler (referert i Eide \&{} Eide, 2008). Studien til McCreaddie \&{} Payne (2011)
viser at det kan være risikabelt for en pasient å bruke humor for å uttrykke
følelser, da sykepleier ikke alltid oppfattet alvoret. Dette vil kunne påvirke
den hjelpen som pasienten mottar i negativ forstand. Det er viktig at vi som
sykepleiere klarer å se alvoret bak spøken. Roaldsen et al. (2015) viser et
eksempel hvor en pasient reagerte med latter da hun mottok sin diagnose. Dette
var hennes måte å håndtere situasjonen og uttrykke følelsene på. Hun var
likevel klar over at sykepleierne kunne oppfatte henne som gal siden hun ikke
gråt.  En alvorlig sykdom som kreft kan være altoppslukende, og man kan lett
bli fokusert og opptatt av egne problemer og lidelser. Dette kan medføre at
evnen til å forholde seg åpen til andre blir redusert. Vi har tidligere sett at
humor kan være kontaktskapende, og i den sammenhengen kan humor være til hjelp
ved at det åpner opp og letter på stemningen. Det kan bidra til bedre kontakt
med andre, og gjøre det lettere å snakke om det som er vanskelig (Eide \&{} Eide,
2008).  Det kan være ubehagelig å gå inn i vanskelige samtaler. Vi kan være
usikre på hva vi skal si eller gjøre, eller kanskje vi er redde for hvordan
pasienten opplever oss. Reitan (2008: 81) mener at det må være rom for dette.
Vi synes ikke at det bør gjøres som i studien til Dean \&{} Major (2007), hvor
sykepleiere brukte humor for å distansere seg fra pasienten, for å slippe å gå
inn i samtalen. Dette anser vi å være en negativ bruk av humoren. Vi tror at
dette kan føre til en dårligere relasjon mellom sykepleier og pasient. Det er
viktig at sykepleiere tørr å møte vanskelige situasjoner, men samtidig være
ærlige om ubehaget.  En positiv side ved bruk av humor er at det kan føre til
tillit mellom pasient og sykepleier. Dette kommer frem i studien til Tanay et
al. (2013) hvor pasientene hadde lettere for å åpne seg og snakke om de
vanskelige tingene dersom humor ble brukt i samtalene. Dette understrekes av
Arnold \&{} Boggs som i tillegg påpeker at det kan redusere vanskelige følelser og
lette på stemningen (referert i Eide \&{} Eide 2008). Videre i Tanay et al. (2013)
ser vi at sykepleierne mente at humor fikk pasientene til å føle seg mer
avslappet og hjemme.

\subsubsection{Individuelle hensyn}

I følge Arnold \&{} Boggs er det ingen faste regler å forholde seg til ved bruk av
humor (referert i Eide \&{} Eide, 2008). Både Arnold \&{} Boggs og Bjørk Breievne
(2011) legger allikevel vekt på at det er forskjellige hensyn en må ta. Dette
synes også vi er viktig å ha fokus på, slik at humoren ikke blir fornærmende
eller får motsatt effekt enn planlagt. Arnold \&{} Boggs mener det er viktig at
man først vurderer personen og situasjonen man befinner seg i (referet i Eide \&
Eide 2008). I Bjørk \&{} Breievne (2011) kommer det frem at humor er personlig og
individuelt. I Dean \&{} Gregory (2005) pekes det også på at man skal ta hensyn
til ulike faktorer. Noe av det som nevnes her er etnisitet, kjønn, stressnivå,
pasientens personlighet og situasjon. En sykepleier sier i et intervju:
\textquote[Dean \&{} Gregory, 2005: 5]{humour
is like individuals, everyone's different.} Et annet
eksempel i artikkelen viser at humor ofte blir dårlig mottatt av pasientene i
situasjoner hvor pasientens tilstand forverres. Dette tror vi kommer av at
pasienten i en slik situasjon kan oppleve mye angst i forhold til det som
skjer, og frykt for fremtiden. Wist (2002) peker i denne sammenheng på at man
kan la pasienten bruke humor omkring sin situasjon, men at man som sykepleier
er mer tilbakeholden med egne humoristiske bemerkninger. Som man kan se av
eksemplene passer det seg ikke alltid med humor i profesjonelle relasjoner.

Noen typer humor kan oppleves krenkende, noen eksempler her kan være sarkasme
og ironi. Det er viktig at man er sikker på at pasienten setter pris på denne
type humor (Eide \&{} Eide, 2008: 246--247). Humor mellom pasient og sykepleier
skal kunne deles, man skal kunne le sammen og ikke av (Wist, 2002: 166).

Vi tror at både pasient og sykepleier kan gå over grensen i forhold til hva som
regnes som akseptabel ved bruk av humor. I Dean \&{} Gregory (2015) nevner
pasientene at det er flere situasjoner der de forstår at de hadde gått for
langt, og at humor ikke var passende. Videre kan vi se at de hadde
vanskeligheter med å svare på spørsmål om når og hvordan de brukte humor. I
Tanay et al. (2013), svarte flere av deltakerne at det ikke var bevisst, men at
det var noe som oppstod spontant. Andre igjen svarte at humorbruken var godt
gjennomtenkt.  Humor er som tidligere sagt forskjellig fra person til person,
og det er derfor varierende hvordan den oppfattes (Bjørk \&{} Breievne, 2011). For
å bevare relasjonen mellom pasient og sykepleier, mener deltakerne i McCreaddie
\&{} Payne (2011) at man bør vurdere situasjonen og se etter tegn fra den andre
før man bruker humor. Videre viser denne artikkelen at pasientene syntes det
var positivt at sykepleierne var muntre og glade, og at de brukte humor. Dette
er noe vi også har erfart. Dersom man møter pasientene med et smil og en
positiv innstilling får man bedre respons. Det var enighet blant pasientene om
at de satt pris på humor, selv i de situasjonene enn skulle tro det ikke var
passende. Det ble omtalt positivt dersom sykepleierne tok sjanser og ikke var
redde for å bruke humor (McCreaddie \&{} Payne, 2011). Dette kan være litt
motstridene i forhold til Arnold og Boggs som mener at humor er uegnet før man
kjenner pasienten relativt godt (referert i Eide \&{} Eide, 2008). Dette tolker vi
som at man skal være sikker på pasienten og ikke ta noen sjanser, da det kan
ødelegge relasjonen. Wist (2002) mener også at vi som sykepleiere skal være
tilbakeholdne med egne humoristiske bemerkninger, men samtidig la pasienten
bruke humor omkring sin egen situasjon.

\subsection{Bruk av humor som mestringsstrategi}

\subsubsection{Håndtering av følelser}

Mestring handler om hvordan et menneske håndterer en belastende situasjon
(Reitan, 2006). I vår oppgave vil det si hvordan den kreftsyke håndterer sin
sykdom og livet med sykdommen.  Perioder med kaos og fortvilelse kan kan være
utfordrende, og det kan være vanskelig ikke å miste seg selv. I Roaldsen et al.
(2015) forteller pasienter at det i slike perioder var særlig viktig å håndtere
følelsene og skape en balanse slik at sykdommen ikke tok opp all plass. De
kunne på den måten distansere seg fra de negative tankene og skape et pusterom
fra situasjonen de befant seg i. Under et kreftforløp er det viktig å prøve å
skape en levelig situasjon, og få en opplevelse av mening og håp. I artikkelen
til Roaldsen et al. (2015) brukte pasientene bevisst humor i forsøket på å
akseptere sykdommen slik at det var lettere å leve med den. Et eksempel som
beskriver dette var en pasient som fortalte at han satte mer pris på livet nå
enn før. Han kunne ofte mimre tilbake til hendelser som han da hadde opplevd
som forferdelige, men som han nå kunne le av, sett i sammenheng med situasjonen
han nå befant seg i. I følge Eide \&{} Eide (2008) kan humor hjelpe pasienten til
forsoning med den situasjonen man befinner seg i, om så bare for en kort
periode.

Flere pasienter forteller at deres bruk av humor var påvirket av graden stress
de erfarte. Man kunne kanskje tro at jo mer stress en pasient opplevde, desto
mindre humor ville de bruke. Studien til Roaldsen et al. (2015) viste et
eksempel av det motsatte hvor pasientene fortalte at de brukte mer humor dersom
stressnivået var veldig høyt. Fordi nivået av stress føltes så uoverkommelig
var dette deres måte å takle situasjonen på. Reitan (2006) beskriver denne
måten å håndtere stress på som en form for forsvarsmekanisme hvor man egentlig
bare skyver bort problemene. Selv om dette ikke var en problemløsende
håndtering fortalte pasientene at de ved å unngå at de tunge tankene og
følelsene fikk komme frem, var det enklere for dem å håndtere omstendighetene
rundt. Som Heggen (2008) påpeker er ikke humor en løsning på problemet, men man
tar i bruk de ressursene man har for å gjøre ting litt mer levelig. Vi har et
uttrykk som heter “en god latter forlenger livet”. En pasient i studien til
Roaldsen et al. (2015) beskriver det nettopp slik. Hun fortale at for henne
fungerte humor som medisin og en beskyttelse mot sin sorg og tristhet
forårsaket av sykdommen.

\subsubsection{Tilbakevennende humor}

I artikkelen til Roaldsen et al., (2015) snakket pasientene om humor som
forsvant og humor som returnerte. Selv om noen pasienter brukte mer humor i
stressende situasjoner, fortalte andre at graden av humor kunne variere gjennom
sykdomsforløpet, og at den kunne forsvinne helt. I perioder hvor hverdagen var
preget av usikkerhet rundt fremtiden og omstendigheter rundt familien kunne
hverdagen bli en så stor psykisk påkjenning at mottakeligheten for humor kunne
forsvinne helt. Når situasjonen så bedret seg opplevde de at humoren kom
tilbake, men at den var endret. Humoren var da gjerne av den mørkere typen som
galgenhumor og ironi.  Dette er en form for mestringsstrategi som man kan se
hos pasienter som har det vanskelig. Man ser gjerne at pasientene spøker bort
alvoret som en måte å beskytte seg selv, og for å dekke over hvor vanskelig de
egentlig har det (Wist, 2002). En pasient forteller om hvordan hun følte det
når hun mistet håret som følge av kreftbehandlingen. I denne situasjonen
fortale hun at kroppen føltes både som en trussel og en støtte. For henne var
det å miste håret det verst tenkelige som kunne skje, og hun følte at bruk av
metaforer og selvironi var det eneste som kunne hjelpe henne å mestre
følelsene. Ved å beskrive seg som “the naked rat” og “Dumbo” satte hun opp en
beskyttelse for å ikke gå inn i en dyp sorg (Roaldsen et al. 2015).
